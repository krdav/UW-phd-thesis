\chapter{Aspartate sensor supplementary figures}

\begin{figure}[ht!]
    \centering
    \fbox{\includegraphics[width=0.7\linewidth]{figures/chap3/Fig1S1.pdf}}
    \caption[Aspartate specificity and excitation/emission spectra.]{
    (A) Switching specificity of the iGluSnFR3 precursor from glutamate to aspartate using S72X library (left) and S72P, S27X library (right).
    Titrations with aspartate (solid lines) and glutamate (dashed lines) in bacterial lysate.
    (B) Excitation and emission spectra of jAspSnFR3-mRuby3.
    Left, 1-photon spectra.
    Excitation wavelength was varied from 400 nm to 520 nm (7.5 nm bandpass) while observing emission at 535 nm (10 nm bandpass).
    Emission wavelength was varied from 535 nm to 600 nm (10 nm bandpass) while exciting at 510 nm (7.5 nm bandpass).
    Fluorescence was measured both in the absence (dashed lines) and presence of 10 mM aspartate (solid lines).
    Right, 2-photon cross-sections, also ± 10 mM aspartate, with an overlay of calculated $\Delta$F/F (green).
    Vertical bar indicates 1040 nm.
    }
    \label{ch3:figsupp:f1S1}
\end{figure}

\begin{figure}[ht]
    \centering
    \fbox{\includegraphics[width=0.7\linewidth]{figures/chap3/Fig1S2.pdf}}
    \caption[Decoy, temperature and pH sensitivity.]{
    (A) jAspSnFR3-mRuby3 does not appreciably change its green fluorescence in response to other amino acids (alanine, phenylalanine, glycine, histidine (red line), isoleucine, leucine, methionine, proline, glutamine, arginine, serine, threonine, valine, or tryptophan).
    Insert with aspartate in black and glutamate/asparagine in grey for comparison.
    Ex. 485 nm (20 nm bandpass), Em. 535 nm (20 nm bandpass), 0.2 $\mu$M purified protein in PBS.
    (B) jAspSnFR3-mRuby3 does not respond to other decoys: citrate, lactate, pyruvate, malate, alpha-ketoglutarate, cis-aconitate, succinate, fumarate, or oxaloacetate (orange squares); nor to relevant pharmacological treatments: rotenone (green squares) or metformin.
    The small increase in fluorescence from rotenone is likely due to the scattering of a visibly turbid solution; rotenone has very low solubility in water.
    Ex. 485 nm (20 nm bandpass), Em. 535 nm (20 nm bandpass).
    (C) jAspSnFR3-mRuby3 is not adversely affected by temperature.
    Fluorescence as a function of aspartate titration at 23°C (light grey), 30°C (medium grey), and 37°C (black).
    Error bars are standard deviation of three technical replicates.
    (D) pH sensitivity of jAspSnFR3-mRuby3 (green component).
    Ex 485 nm (5 nm bp), Em 515 nm (10 nm bp).
    Error bars are standard deviation of 5 technical replicates.
    Solid line is with 3 mM aspartate, dashed line is without aspartate.
    }
    \label{ch3:figsupp:f1S2}
\end{figure}

\begin{figure}[ht]
    \centering
    \fbox{\includegraphics[width=0.8\linewidth]{figures/chap3/Fig1S3.pdf}}
    \caption[Stopped-flow kinetics.]{
    (A) Stopped-flow kinetics of jAspSnFR3 using different aspartate concentrations for injection.
    Measurements were performed at a frequency of 1 per msec. and indicated by dots.
    To each time-series an exponential function was fit, shown as a solid line with matching color.
    (B) $k_{obs}$ as a function of aspartate concentration shown for two independent stopped-flow experiments.
    The line represents the linear function, fitted to the linear range to extract the kinetic rates.
    }
    \label{ch3:figsupp:f1S3}
\end{figure}

\begin{figure}[ht]
    \centering
    \fbox{\includegraphics[width=0.7\linewidth]{figures/chap3/Fig1S4.pdf}}
    \caption[mRuby3 interaction with histidine tag.]{
    (A) jAspSnFR3-mRuby3 shows increased red fluorescence at millimolar concentrations of all amino acids amino acids (alanine, phenylalanine, glycine, histidine, isoleucine, leucine, methionine, proline, glutamine, arginine, serine, threonine, valine, or tryptophan), with apparent responses to histidine at 100 $\mu$M (red line).
    (B) Increased red fluorescence of mRuby3 in response to histidine requires a C-terminal histidine tag.
    (C) Increased red fluorescence of mRuby3 in response to millimolar concentrations of aspartate requires a C-terminal histidine tag.
    (D) mRuby, with or without histidine tag, does not increase in fluorescence upon treatment with amino acid related compound gamma-aminobutyric acid (GABA).
    For all plots Ex. 555 nm (20 nm bandpass), Em. 600 nm (20 nm bandpass).
    }
    \label{ch3:figsupp:f1S4}
\end{figure}

\begin{figure}[ht]
    \centering
    \fbox{\includegraphics[width=0.98\linewidth]{figures/chap3/Fig2S1.pdf}}
    \caption[Rotenone titration in different cell lines.]{
    jAspSnFR3 temporal response after rotenone treatment.
    (A) HT1080 cells using nuclear RFP to normalize the jAspSnFR3 signal, treated with a rotenone titration.
    (B) HT1080 cells using an RFP fused to jAspSnFR3 (jAspSnFR3-mRuby3) for normalization, treated with a rotenone titration.
    (C) HT1080 cells treated with 100 nM rotenone at the start of the experiment (0 h) and then rescued with pyruvate at start, 22 h or never.
    (D) Comparison between the steady-state signal of (A) and (B) with a linear regression shown as a red dashed line to show that nuclear RFP and RFP fusion normalizations are equivalent.
    (E) HEK293t cells using an RFP fused to jAspSnFR3 for normalization, treated with a rotenone titration.
    For plots (A), (B), (C) and (E) markers indicate the average using available well replicates and are superimposed on a bootstrapped 95\% confidence interval colored using the same color code as the markers.
    For plot (D) markers indicate the average using available well replicates and errorbars are drawn as +/- the standard deviation of the replicates.
    Grey dashed lines indicate the time of treatment.
    Orange dashed line in panel (C) indicates time of pyruvate addition.
    AU, arbitrary unit.
    }
    \label{ch3:figsupp:f2S1}
\end{figure}

\begin{figure}[ht]
    \centering
    \fbox{\includegraphics[width=0.9\linewidth]{figures/chap3/Fig2S2.pdf}}
    \caption[Plots related to glutamine limitation.]{
    (A) Nuclei count over time for conditions displayed in figure \ref{ch3:fig:Fig2}, panel E.
    (B) Cell confluency over time for conditions displayed in figure \ref{ch3:fig:Fig2}, panel E.
    (C) H1299 cells changed into media with a titration of glutamine with or without 1 mM asparagine.
    Identical to figure \ref{ch3:fig:Fig2}, panel H but with fewer glutamine concentrations and more well replicates.
    }
    \label{ch3:figsupp:f2S2}
\end{figure}

\begin{figure}[ht]
    \centering
    \fbox{\includegraphics[width=0.98\linewidth]{figures/chap3/Fig3S1.pdf}}
    \caption[jAspSnFR3 signal does not correlate with glutamate concentration.]{
    RFP normalized jAspSnFR3 signal, following various perturbations to live cells, is not correlated with the LCMS measured intracellular glutamate concentration.
    Datapoints are fitted to a local linear regression, shown by the black line, otherwise, these plots are identical to those in figure \ref{ch3:fig:Fig3}.
    AU, arbitrary unit.
    }
    \label{ch3:figsupp:f3S1}
\end{figure}







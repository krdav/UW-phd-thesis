


\documentclass[proquest]{uwthesis}[2020/02/24]
\usepackage{graphicx}
\usepackage{caption}
\usepackage{listings}
\usepackage{listings}
\lstset{
  basicstyle=\ttfamily,
  columns=fullflexible,
  frame=single,
  breaklines=true,
  %postbreak=\mbox{\textcolor{red}{$\hookrightarrow$}\space},
}
\usepackage{subcaption}
\usepackage{blindtext}
\usepackage{hyperref}
\usepackage[section]{placeins} % force figures into place
\usepackage{setspace} % to decrease line spacing locally
\usepackage{amsmath} % for better math
\usepackage[shortcuts]{extdash} % for dash without line break
% Setup fnsymbols with both numbers and symbols for author list footnotes:
\usepackage[symbol*]{footmisc}
\DefineFNsymbolsTM{author_list}{
  1 1
  2 2
  3 3
  \textdagger    \dagger
  \textasteriskcentered *
  \textdaggerdbl \ddagger
  \textsection   \mathsection
  \textbardbl    \|%
  \textparagraph \mathparagraph
}
\setfnsymbol{author_list}

\hypersetup{
    colorlinks=true,
    linkcolor=blue,
    filecolor=magenta,      
    urlcolor=cyan,
    pdfpagemode=FullScreen,
}

% With so many floats, vertical alignment becomes stupid:
\raggedbottom


\setcounter{tocdepth}{1}  % Print the chapter and sections to the toc
 

% ==========   Local defs and mods
%

% --- sample stuff only -----
% These format the sample code in this document

\usepackage{alltt}  % 
\newenvironment{demo}
  {\begin{alltt}\leftskip3em
     \def\\{\ttfamily\char`\\}%
     \def\{{\ttfamily\char`\{}%
     \def\}{\ttfamily\char`\}}}
  {\end{alltt}}
 
% metafont font.  If logo not available, use the second form
%
% \font\mffont=logosl10 scaled\magstep1
\let\mffont=\sf
% --- end-of-sample-stuff ---





%%%% Shortcuts %%%%%

\newcommand{\natlog}{\text{ln}}
\newcommand{\Asn}{\text{Asn}}
\newcommand{\UAsn}{\text{U-\textsuperscript{13}C Asn}}
\newcommand{\Flin}{J_{\text{in}}}
\newcommand{\Flout}{J_{\text{out}}}
\newcommand{\Flsyn}{J_{\text{syn}}}
\newcommand{\Flprot}{J_{\text{prot}}}
\newcommand{\hCi}{\textsuperscript{13}C}
\newcommand{\hNi}{\textsuperscript{15}N}
\newcommand{\NAD}{NAD\textsuperscript{+}}
\newcommand{\NADP}{NADP\textsuperscript{+}}





 


\begin{document}
 
% ==========   Preliminary pages
%
% ( revised 2012 for electronic submission )
%

\prelimpages
 
%
% ----- copyright and title pages
%
\Title{Tools to investigate aspartate limited proliferation\\ of cancer cells}
\Author{Kristian Davidsen}
\Year{2024}
\Program{Molecular and cellular biology}

\Chair{Lucas B Sullivan}{}{Fred Hutch Cancer Center, Human Biology}
\Signature{Arvind R Subramaniam}
\Signature{Dana L Miller}

\copyrightpage
\titlepage  


%
% ----- abstract
%





\setcounter{page}{-1}
\abstract{%
Cellular aspartate levels are tightly linked to cell proliferation rate.
But how aspartate limitation decreases proliferation is unknown and could be related to its role as an amino acid as well as a substrate in multiple metabolic reactions.
In this dissertation, the role of each of the metabolic fates of aspartate is interrogated.
Surprisingly, it was not possible to rescue proliferation during aspartate limitation by providing the fates of aspartate, neither were there any evidence of insufficiency for protein synthesis.
It is hypothesized that nucleotide imbalance is responsible and that this would elude efforts to rescue proliferation through complementation.
Nevertheless, these results prompted the development of better tools to measure aspartate and the consequences of aspartate limitation.
Towards that end, an aspartate biosensor was developed and tested, enabling increased throughput of non-invasive aspartate measurements under relevant growth conditions.
Also, a highly accurate method for measuring tRNA aminoacylation using sequencing was developed.
This tRNA-Seq method was tested on human tRNAs with a titration of charge levels, its measurements were determined to be quantitative and with enough accuracy and precision to detect small ($\sim$5\%) effects on tRNA charge.
}
 
%
% ----- contents & etc.
%
\tableofcontents
\listoffigures
\listoftables
 
% Generally, better to write in a way such that no glossary is necessary
%\chapter*{Glossary}      % starred form omits the `chapter x'
\addcontentsline{toc}{chapter}{Glossary}
\thispagestyle{plain}
%
\begin{glossary}
\item[ISR] integrated stress response.
\item[ETC] electron transport chain.
\item[Total cell conc.] the total molar uptake divided by the total increase in cell volume.
\item[TCA] jhjjjj
 
\end{glossary}

% end of the preliminary pages
 
 
 
%
% ==========      Text pages
%

\textpages

\acknowledgments{% \vskip2pc
I would like to acknowledge all the people that I have been working with for the past five and a half years.
My lab mates, current and previous members of the Sullivan lab: 
Jen Crainic,
Ian Engstrom,
Madeleine Hart,
Ayaha Itokawa,
Adam Krol,
Oliver Newsom,
Evan Quon,
David Sokolov,
Anna-Lena Vigil
and PI Lucas Sullivan.
My previous PI Erick Matsen and members of his group:
Arman Bilge,
Will DeWitt,
Amrit Dhar,
Jean Feng,
Eli Harkins,
Andy Magee,
Branden Olson,
Duncan Ralph and
David Shaw.
The members on my committee:
Taran Gujral,
Dana Miller,
Daniel Raftery and
Arvind (Rasi) Subramaniam.
Various collaborators, rotations labs and helpful people:
Kristin Anderson,
Alicia Darnell and the Vander Heiden lab,
Liangcai Gu and the Gu lab,
Jonathan Marvin,
JP McNevin,
Brian Milles,
Vladimir Minin,
Smita Yadav and the Yadav lab.
And my wife Giang Tra Le.

And as prescribed by the ancient cliché in all acknowledgement sections, a dedication or a person for which ``this work could not have been completed without'' must be identified.
Well, I hereby declare that this work is dedicated to all US taxpayers - you paid for it!
}

%»Det værste ved ikke at lave noget er, at man aldrig ved, hvornår man er færdig.«\\
%Storm P



% General introduction to aspartate metabolism,
% starting with the bascis then moving into aspartate limitation
% and the relevant papers:
\chapter{Aspartate in cellular metabolism}
blaa

\section{Introduction}
More blaa

\subsection{Intro subject 1}
qwerty

\subsection{Intro subject 2}
qwerty
\subsection{Details}
qwerty1




Aspartate is typically coupled to NAD+/NADH ratio and thus if one is changed the other one is changed proportionally an vice versa.
Therefore it could be compelling to think that aspartate levels correlate with proliferation only indirectly through the NAD+/NADH ratio.
However, we have evidence against this:
Lucas' previous papers
My NAD+/NADH experiment with 143B SLC1A3 cells
Maddie's paper on SHDB which shows the opposite trend i.e. increased cytoplasmic NAD+/NADH ratio (due to media pyruvate) but are still aspartate limited






 % Aspartate in cellular metabolism

% All the negative data collected throughout the years:
\chapter{Molecular mechanisms of aspartate limitation}
blaa

\section{Introduction}
More blaa




\section{Mitochondrial inhibition induces a stable relationship between aspartate levels and proliferation rate}
%%% Experiments with H1299 and count/metabolites per day
%% Under ETCinhib_timeseries in lab-work







\section{Quantifying the metabolic fates of aspartate}


\begin{figure}
     \centering
     \begin{subfigure}[b]{0.75\textwidth}
         \centering
         \includegraphics[width=\textwidth]{figures/chap2/flux_143wt.pdf}
         \caption{Media uptake 143B WT}
         \label{fig:ch2:flux_143wt}
     \end{subfigure}
     \begin{subfigure}[b]{0.75\textwidth}
         \centering
         \includegraphics[width=\textwidth]{figures/chap2/flux_143dko.pdf}
         \caption{Media uptake 143B GOT DKO}
         \label{fig:ch2:flux_143dko}
     \end{subfigure}
        \caption[Media amino acid uptake]{
        Amino acid influx/efflux from DMEM.
        For 143B WT in (a), media was spiked with U-\hCi{} Asn (*Asn in the plot) to measure asparagine uptake which, taken together with the labelling fraction, was used to calculate the net asparagine consumption.
        For 143B GOT DKO in (b), cell expressed Glu/Asp transporter SLC1A3 and media was spiked with aspartate to measure its uptake.
        }
\end{figure}



Good correspondence between Asn efflux estimated through labelled Asn in 143B WT (figure \ref{fig:ch2:Asn_flux}) and measured Asn efflux in 143B GOT DKO (figure \ref{fig:ch2:flux_143dko}).
\begin{figure}
     \centering
     \hspace{0.05\textwidth}
     \begin{subfigure}[b]{0.35\textwidth}
         \centering
         \includegraphics[width=\textwidth]{figures/chap2/asn_Jprot.pdf}
         \caption{Flux diagram}
         \label{fig:ch2:asn_Jprot}
     \end{subfigure}
     \hfill
     \begin{subfigure}[b]{0.4\textwidth}
         \centering
         \includegraphics[width=\textwidth]{figures/chap2/Asn_flux.pdf}
         \caption{Asn consumption}
         \label{fig:ch2:Asn_flux}
     \end{subfigure}
     \hspace{0.05\textwidth}
        \caption[Asparagine consumption fluxes]{
        (a) Diagram of asparagine fluxes in a cell.
        Net influx of U-\hCi{} labelled Asn ($\Flin$), net efflux of unlabelled Asn ($\Flout$), net deposition of Asn into protein ($\Flprot$) and Asn synthesis from Asp ($\Flsyn$).
        Asn symbolized by green and U-\hCi{} Asn symbolized by blue.
        (b) Measured asparagine consumption fluxes in 143B cells using the media uptake data from figure \ref{fig:ch2:flux_143wt}.
        Asn is consumed into protein synthesis (Protein) and leaked into the media (Efflux).
        Efflux is calculated assuming no Asn is provided in the media.
        }
\end{figure}



Most of cellular biomass is amino acids \cite{Hosios2016-us}

\begin{figure}
    \centering
    \includegraphics[width=0.80\textwidth]{figures/chap2/ah_cell_comp.pdf}
    \caption[Amino acid and pyrimidine total cell concentration]{
    Total amino acids and pyrimidines liberated by acid hydrolysis of 143B and H1299 cells, normalized by total cell volume (Total cell conc.).
    Gln and Asn is converted during acid hydrolysis to Glu and Asp respectively.
    }
    \label{fig:ch2:ah_cell_comp}
\end{figure}






The total Asp consumption flux expected in DMEM i.e. with Arg but without Asn, is estimated to be 9.9 mM/h.
This is close to Asp uptake in 143B GOT DKO which was measured at 11.4 mM/h (figure \ref{fig:ch2:flux_143dko}).
The difference is directionally as would be expected because total pyrimidine/purine levels underestimate synthesis flux by discounting recycling (figure ]\ref{fig:ch2:pur_tr_ov}) and because 143B GOT DKO maintain a higher intracellular concentration of aspartate due to expression of Glu/Asp transporter SLC1A3.
\begin{figure}
    \centering
    \includegraphics[width=0.36\textwidth]{figures/chap2/asp_fate.pdf}
    \caption[Relative consumption towards each fate of aspartate]{
    Relative consumption towards each fate of aspartate estimated using best estimates from figure \ref{fig:ch2:Asn_flux} and \ref{fig:ch2:ah_cell_comp}.
    Aspartate consumption towards purines was estimated 1.5 times the consumption towards pyrimidines.
    * Asn efflux is calculated based on the assumption that cells are grown in asparagine free media, prior to substantial media conditioning.
    Bars colored blue to indicate that the aspartate fate is a nitrogen donation yielding fumarate.
    }
    \label{fig:ch2:asp_fate}
\end{figure}






\section{Salvage of the metabolic fates of aspartate}

\subsection{Salvage mix fulfills all the metabolic fates of aspartate}

metabolic fates of aspartate i.e. aspartate conversion 


\begin{figure}
    \centering
    \includegraphics[width=0.95\textwidth]{figures/chap2/purine_tracing_overvew.pdf}
    \caption[Purine metabolism \hNi-amide Gln tracing overview]{
    Overview of \hNi-amide Gln label incorporation in \textit{de novo} purine synthesis.
    Label incorporation can be effected by salvage of unlabelled hypoxanthine or adenine and recycling as it appears on the overview.
    }
    \label{fig:ch2:pur_tr_ov}
\end{figure}





\begin{figure}
    \centering
    \includegraphics[width=0.95\textwidth]{figures/chap2/gln_lab_tranfr.pdf}
    \caption[Gln amide to alpha \hNi{} transfer]{
    Gln \hNi{} on the amide nitrogen can transfer to the alpha nitrogen in H1299 cells but not in 143B cells.
    Upper left diagram shows how Gln amide and alpha nitrogen labels can transfer.
    Transfer of amide labelled nitrogen can be achieved by glutaminase catalyzed release of labelled ammonia and its subsequent use as a substrate in the conversion of alpha-ketoglutarate (aKG) to Glu alpha-\hNi by glutamate dehydrogenase.
    The signature of glutamine synthetase activity is the appearance of doubly labelled Gln.
    Upper right diagram shows how Gln amide and alpha nitrogen labels are transferred to downstream metabolites Glu, Asp and Asn.
    Lower panel shows the nitrogen isotopologue distribution of Gln, Glu, Asp and Asn in 143B and H1299 at steady-state.
    The alpha nitrogen label is frequently lost in Glu and downstream, presumably due to transaminase catalyzed exchange with unlabelled amino groups on amino acids such as leucine, isoleucine, valine etc.
    The amide nitrogen label is partially transferred to the alpha position in H1299, but not in 143B, also indicated by the doubly labelled Asn.
    }
    \label{fig:ch2:gln_lab_tranfr}
\end{figure}


\begin{figure}
    \centering
    \includegraphics[width=0.98\textwidth]{figures/chap2/sal_frac_pur.pdf}
    \caption[Salvage into purines]{
    ggg
    }
    \label{fig:ch2:sal_frac_pur}
\end{figure}


\begin{figure}
    \centering
    \includegraphics[width=0.8\textwidth]{figures/chap2/sal_frac_pyr-asn.pdf}
    \caption[Salvage into asparagine and pyrimidines]{
    ggg
    }
    \label{fig:ch2:sal_frac_pyr-asn}
\end{figure}









\begin{figure}
    \centering
    \includegraphics[width=0.95\textwidth]{figures/chap2/sal_frac_conc.pdf}
    \caption[Salvage as a function of Urd/Hpx concentration]{
    Fraction of purines (GDP, GMP, ADP and AMP) and pyrimidines (UDP, UMP, CDP and CMP) derived from salvage when 143B or H1299 cells are cultured in increasing concentrations of uridine/hypoxanthine.
    }
    \label{fig:ch2:sal_frac_conc}
\end{figure}







\section{Aspartate to proliferation curves}

%%% Experiments with individual components under "ETCrescue" folder in lab-work

%%% Experiments with H1299 (Metformin and Rotenone) and HT1080




\section{Integrated stress response}

Overlap between perturbations eliciting ISR by GCN2 and HRI \cite{Taniuchi2016-nc}



\begin{figure}
     \centering
     \begin{subfigure}[b]{0.49\textwidth}
         \includegraphics[width=\textwidth]{figures/chap2/143B_ETCinhib_ATF4rep_low.pdf}
         \caption{ggg}
         \label{fig:ch2:143B_ETCinhib_ATF4rep_low}
     \end{subfigure}
     \hfill
     \begin{subfigure}[b]{0.49\textwidth}
         \includegraphics[width=\textwidth]{figures/chap2/HT1080_ETCinhib_ATF4rep_high.pdf}
         \caption{ggg}
         \label{fig:ch2:HT1080_ETCinhib_ATF4rep_high}
     \end{subfigure}
     \hfill
     \begin{subfigure}[b]{0.4\textwidth}
         \includegraphics[width=\textwidth]{figures/chap2/143B_Atp_ATF4rep.pdf}
         \caption{ggg}
         \label{fig:ch2:143B_Atp_ATF4rep}
     \end{subfigure}
     \hspace{0.06\textwidth}
     \begin{subfigure}[b]{0.4\textwidth}
         \includegraphics[width=\textwidth]{figures/chap2/HT1080_GOT_DKO_ETCinhib_ATF4rep.pdf}
         \caption{ggg}
         \label{fig:ch2:HT1080_GOT_DKO_ETCinhib_ATF4rep}
     \end{subfigure}
        \caption[ggg]{
        gggg
        }
        \label{fig:ch2:ISR}
\end{figure}











\section{Methods and Materials}

\subsection{Cell culture}
Cell lines were acquired from ATCC (143B, H1299, HT1080) and tested to be free from mycoplasma (MycoProbe, R\&D Systems).
Cells were maintained in Dulbecco’s Modified Eagle’s Medium (DMEM) (Gibco, 50-003-PB) supplemented with 3.7 g/L sodium bicarbonate (Sigma, S6297), 10\% fetal bovine serum (FBS) (Gibco, 26140079) and 1\% penicillin-streptomycin solution (Sigma, P4333).
Cells were incubated in a humidified incubator at 37°C with 5\% CO2.


\subsection{Western blots}
Protein lysates were harvested in RIPA buffer (Sigma, R0278) supplemented with Halt protease and phosphatase inhibitor cocktail (Fisher, PI78443) and 5 mM EDTA.
Protein concentration was determined using a bicinchoninic acid assay (Fisher, 23225) using bovine serum albumin (BSA) as a protein standard.
Equal amounts of protein were added to LDS sample buffer (ThermoFisher, B0008) and 5\% 2-Mercaptoethanol (Sigma, M3148), denatured at 95°C for 5 min, loaded onto 4–12\% SDS-polyacrylamide gels (Invitrogen, NW04122BOX) along with a prestained protein ladder (ThermoFisher, 26616) and run 35 min at 180V in MES buffer (ThermoFisher, B000202).
Proteins were dry transferred onto a 0.22 mm nitrocellulose membranes using the iBlot2 device (ThermoFisher, IB21001) with the P0 system setting and associated transfer stacks (Fisher, IB23001).
Membranes were blocked with 5\% bovine serum albumin; Sigma, A4503 in tris-buffered saline with 0.1\% Tween-20 (TBS-T) and incubated at 4°C overnight with the following antibodies:
anti-ATF4 (Cell Signaling, 11815S, 1:500),
anti-ASNS (Cell Signaling, 92479T, 1:1000),
anti-Phospho-eIF2alpha (Cell Signaling, 3398S, 1:500),
anti-eIF2alpha (Cell Signaling, 5324S, 1:500),
anti-Phospho-GCN2 (Cell Signaling, 94668S, 1:500),
anti-GCN2 (Cell Signaling, 3302S, 1:500),
anti-OMA1 (Cell Signaling, 95473, 1:1000),
anti-GADD34 (Proteintech, 10449-1-AP, 1:500),
anti-DARS2 (Proteintech, 13807-1-AP, 1:500),
anti-HRI (Sigma, HPA016496, 1:1000),
anti-FLAG (Sigma, F1804; 1:1000),
anti-SLC1A3 (Genetex, GTX20262; 1:500),
anti-GOT2 (Proteintech, 14800–1-AP, 1:750),
anti-GOT1 (Cell Signaling, 34423 S, 1:1000),
anti-Vinculin (Sigma, SAB4200729; 1:10,000) and anti-Tubulin (Sigma, T6199; 1:10,000).
Membranes were washed with TBS-T and the following secondary antibodies were added in blocking buffer: 800CW Goat anti-Mouse IgG (LiCOR, 926–32210; 1:15,000), 680RD Goat anti-Rabbit IgG (LiCOR, 926–68071; 1:15,000) and incubated for 1 hour.
Membranes were washed with TBS-T, incubated for 10 min in TBS-T, washed in deionized water and imaged on a LiCOR Odyssey Near-Infrared imaging system.


\subsection{Proliferation assays}
Cells were trypsinized (Corning, 25,051 CI), resuspended in seeding media, counted (Beckman Coulter Counter Multisizer 4) and seeded overnight onto 6/12/24-well dishes (Corning, 3516;3513:3524) with an initial seeding density of 10,000 cells/mL and a volume of 4, 2 and 1 mL, respectively.
After overnight incubation, 3–6 wells were counted for a starting cell count at the time of treatment.
Treatment was initiated either by media switch or by spike-in of drug/metabolite from a 20-50x stock.
Experiments were conducted in (both for seeding and treatment) DMEM without pyruvate (Corning 50–013-PB) supplemented with 3.7 g/L sodium bicarbonate 10\% dialyzed fetal bovine serum (FBS) (Sigma, F0392) and 1\% penicillin-streptomycin solution, with or without sodium pyruvate (Pyr) (Sigma, P8574),
2-ketobutyric acid (AKB) (Sigma, K401),
aspartate (Asp) (Sigma, A7219),
asparagine (Asn) (Sigma, A7094),
uridine (Urd) (Sigma, U3003), hypoxanthine (Cayman Chemical, 22254),
adenine (Ade) (Sigma, A2786),
guanine (Gua) (Sigma, 51030) or sodium formate (Sigma, 71539) with concentration noted when relevant.
Drug treatments included rotenone (Sigma, R8875),
metformin (Sigma, D150959),
atpenin A5 (Cayman Chemical, 11898; AdipoGen, AG-CN2-0110; Abcam, ab144194; or Enzo Life Sciences, ALX-380–313),
doxycycline hydrochloride (Sigma, D3447),
antimycin A (Sigma, A8674),
oligomycin A (Sigma, 495455),
GCN2iB (MedChemExpress, HY-112654),
FCCP (Cayman Chemical, 15218-10),
BAM15 (Cayman Chemical, 17811),
UCPH (HelloBio, HB0630) and DMSO vehicle (Sigma, D2650).
Cells were incubated in a humidified incubator at 37°C with 5\% CO2, then counted after 4–6 days.
Proliferation rate was reported as doublings per day and determined using the time and fold count difference between the starting and final counts and assuming a constant proliferation rate throughout the assay.


\subsection{Generation of nuclear RFP cell lines}
Nuclear RFP cell lines were generated using 1e5 transducing units of EF1A-nuclear RFP lentivirus (Cellomics Technology, PLV-10205-50) by spinfection.
Cells were seeded at 50\% confluency on 6 well dishes, lentivirus was added to fresh media with 8 µg/µL polybrene, then added to cells and followed by centrifugation (900g, 90 mins, 30°C).
Two days after infection, cells were sorted for high RFP expression using fluorescence-activated cell sorting (FACS).
High RFP cells were then expanded and single-cell cloned by limiting dilution, plating 0.5 cells/well on a 96 well plate.
Plates were then screened for RFP expression and localization using Incucyte S3 (Sartorius) and a suitable clone chosen, expanded, and used for all subsequent experiments.


\subsection{Incucyte measurements}
Proliferation assay using Incucyte



\subsection{Lentiviral production and stable cell line generation}
The following plasmids were obtained: pLenti6.3-V5 DEST\_SLC1A3 (DNASU Plasmid Repository), ATF4 reporter pXG237 (Addgene, 141281), pLHCX-gpASNase1 (Addgene, 121526) and pDONR221\_EGFP (Addgene, 25899).
ASNS was cloned by PCR from HEK293T reverse transcribed mRNA.
Genes were first cloned into entry vector pENTR1A (Fisher, A10462) using NEBuilder HiFI DNA Assembly Cloning Kit (New England BioLabs, E2621).
These donor constructs were then used to transfer their insert into destination vectors: pLX304-CMV-Blast (Addgene, 25890) or pLenti-CMV-Hygro (w117-1) (Addgene, 17454 a gift from Eric Campeau \& Paul Kaufman) using LR Clonase II (Fisher, 11791100).
Each plasmid sequence was verified by whole plasmid sequencing (Plasmidsaurus).
Lentivirus was generated by co-transfection of HEK293T cells with destination vector plasmid DNA and the packaging plasmids pMDLg/pRRE (Addgene, 12251), pRSV-Rev, (Addgene, 12253) and pMD2.G (Addgene, 12259) using FuGENE transfection reagent (Fisher, PRE2693) in DMEM (Fisher, MT10017CV) without FBS or penicillin-streptomycin.
The supernatant containing lentiviral particles was filtered through a 0.45 µM membrane (Fisher, 9720514) and was supplemented with 8 µg/µL polybrene (Sigma, TR-1003-G) prior to infection.
For infection, cells were seeded at 50\% confluency in 6 well dishes and centrifuged with lentivirus (900g, 90 mins, 30°C).
After 24 hours the media was replaced with fresh media and after 48 hours cells were treated with either 1 µg/mL blasticidin (Fisher, R21001) or 150 µg/mL hygromycin (Sigma, H7772-1G) and maintained in selection media until all uninfected control cells died.
After selection, cells were expanded and single-cell cloned by limiting dilution, plating 0.5 cells/well using 96 well plates.
These clones were expanded and screened by either western blot or presence of GFP or RFP signal using Incucyte S3 (Sartorius) to validate expression.
From this a single clone was chosen, expanded and used for all subsequent experiments.


\subsection{Generation of knockout cells}
Protocol and guide RNA generation was identical to that described in Hart et al. \cite{Hart2023-gp}.
Briefly, three chemically synthesized 2'-O-methyl 3’phosphorothioate-modified single guide RNA (sgRNA) sequences targeting the gene of interest were purchased (Synthego; table \ref{tab:ch2:guides}).
A pool of all three sgRNAs (or all six for GOT1/GOT2 double knockout) were resuspended in nuclease-free water, combined with SF buffer (Lonza, V4XC-2032), and sNLS-spCas9 (Aldevron, 9212).
200,000 H1299 cells were resuspended in the resulting solution containing ribonucleoprotein complexes (RNPs) and electroporated using a 4D-Nucleofector (Amaxa, Lonza).
Nucleofected cells were then expanded and single-cell cloned by limiting dilution by plating 0.5 cells/well in a 96 well plate.
Gene knockout was confirmed using western blots.

\begin{spacing}{1}
\begin{table}[ht]
\caption{\label{tab:ch2:guides}CRISPR guides.}
\begin{tabular}{|l|l|}
\hline
Gene & sgRNA   sequence (5’-3’) \\
\hline
GOT1 & \begin{tabular}[c]{@{}l@{}}\texttt{CAGUCAUCCGUGCGAUAUGC}\\\texttt{GCACGGAUGACUGCCAUCCC}\\\texttt{CGAUCUUCUCCAUCUGGGAA}\end{tabular} \\
\hline
GOT2 & \begin{tabular}[c]{@{}l@{}}\texttt{UUUCUCAUUUCAGCUCCUGG}\\\texttt{CGGACGCUAGGCAGAACGUA}\\\texttt{UCCUUCCACUGUUCCGGACG}\end{tabular} \\
\hline
OMA1 & \begin{tabular}[c]{@{}l@{}}\texttt{ACACAUUAGCAUCCACCUCA}\\\texttt{GAGUAAAUCAGUGUGACAGG}\\\texttt{GCCAACCCAAGAUGCCAGAA}\end{tabular} \\
\hline
HRI & \begin{tabular}[c]{@{}l@{}}\texttt{GUUUGCAACUGCAAAAGGGA}\\\texttt{UGAUGUUCCAGCAGAAAUCC}\\\texttt{CCAGCACCUUCACUUCCCGU}\end{tabular} \\
\hline
GCN2 & \begin{tabular}[c]{@{}l@{}}\texttt{AAAACUAAAUUGAUUUCAGG}\\\texttt{AGCUCGGUCAUCCUUGGCCA}\\\texttt{GAACUGGCCAAGAAACACUG}\end{tabular} \\
\hline
SLC25A10 & \begin{tabular}[c]{@{}l@{}}\texttt{GCAUCUGCAGACGCAGCAGG}\\\texttt{GCAACACCUUCUCGUGGAAG}\\\texttt{GAAGCUGCGCAUGACGGGCA}\end{tabular} \\
\hline
DARS2 & \begin{tabular}[c]{@{}l@{}}\texttt{ACAUAAAAUCUUCUUCACAG}\\\texttt{UGGUUAAGUCAGCUGUACAG}\\\texttt{GUGGAUGGAUUCAGUACCGA}\end{tabular} \\
\hline
\end{tabular}
\end{table}
\end{spacing}

\subsection{Polar metabolite extraction}
For polar metabolite extraction, a plate was move to ice and the media was thoroughly aspirated.
Wells were washed thrice with cold saline (Fisher, 23293184), 1 mL 80\% HPLC grade methanol in HPLC grade water was added, cells were scraped with the back of a P1000 pipet tip and transferred to Eppendorf tubes.
Tubes were centrifuged (17,000g, 15 mins, 4°C) and a fraction of the supernatant containing polar metabolites was transferred to a new centrifuge tube and placed in a centrivap until dry.
The fraction of supernatant transferred was adjusted to correspond to that extracted from a 1 µL cell volume e.g. 50\% was transferred if the total cell volume extracted from was 2 µL.
The total cell volume extracted from was determined by counting cells on a parallel plate using a coulter counter.
Dried samples were reconstituted with 40 µL 80\% HPLC grade methanol, containing internal standards if appropriate, and transferred to vials for measurement by LCMS.

\subsection{Media metabolite extraction}
For media metabolite extraction, 10 µL media was sampled, added to 990 µL 80\% HPLC grade methanol in HPLC grade water and incubated at -20°C for 30 min or until ready.
Tubes were centrifuged (17,000g, 15 mins, 4°C) and 400 µL of the supernatant containing media metabolites was transferred to a new centrifuge tube and placed in a centrivap until dry.
Dried samples were reconstituted with 40 µL 80\% HPLC grade methanol, containing internal standards if appropriate, and transferred to vials for measurement by LCMS.


\subsection{Absolute quantification by isotope dilution}
Dried samples were reconstituted with 40 µL 80\% HPLC grade methanol containing 5 µM U-\hCi, U-\hNi{} labelled canonical amino acid mix (Cambridge Isotope Laboratories, MSK-CAA-1) and transferred to vials for measurement by LCMS.
For pyrimidine nucleobase/nucleoside quantification a U-\hCi{} internal standard was made by partial hydrolysis (12 h in 6 M HCl at 90°C) of U-\hCi{} spirulina whole cells lyophilized powder (Cambridge Isotope Laboratories, CLM-8400-PK).
The peak area for each compound was divided by its labelled standard to derive the response ratio.
The response ratio was then mapped to a calibration curve to infer the compound concentration in the vial.
The sample concentration was calculated by correcting for each step introducing a dilution, for the intracellular concentrations this included using of the total cell volume.
To make the calibration curves a non-labelled amino acid mixture was made from an analytical amino acid standard without glutamine and asparagine (Sigma, A9906-1ML) and added glutamine (Sigma, 76523-100MG) and asparagine (Sigma, 51363-100MG) to match the concentration of the other amino acids.
For pyrimidine nucleobase/nucleoside quantification this pool was also mixed with equimolar uracil (Sigma, U1128), uridine (Sigma, U3003), 2′-deoxyuridine (Sigma, D5412), thymine (Sigma, T0376), cytosine (Sigma, C3506) and cytidine (Cayman Chemical, 29602).
Using this mix, three replicates of a 12 point 2-fold dilution series was made with a max concentration of 500 µM and a volume per dilution of 40 µL.
These were placed in a centrivap until dry and reconstituted with 40 µL 80\% HPLC grade methanol containing the appropriate isotopic internal standard and transferred to vials for measurement by LCMS.
The peak area for each compound was divided by its labelled standard to derive the response ratio, then the best fitting calibration curves for each compound were chosen among either linear, power or a second-degree polynomial.
Each calibration curve was manually inspected for proper fit and measurements below or above the concentration range of the dilution series were discarded.


\subsection{Liquid Chromatography-Mass Spectrometry (LCMS)}
Metabolite quantitation was performed using a Q Exactive HF-X Hybrid Quadrupole-Orbitrap Mass Spectrometer equipped with an Ion Max API source and H-ESI II probe, coupled to a Vanquish Flex Binary UHPLC system (Thermo Scientific).
Mass calibrations were completed at a minimum of every 5 days in both the positive and negative polarity modes using LTQ Velos ESI Calibration Solution (Pierce).
Polar Samples were chromatographically separated by injecting a sample volume of 1 µL into a SeQuant ZIC-pHILIC Polymeric column (2.1 x 150 mm 5 mM, EMD Millipore).
The flow rate was set to 150 mL/min, autosampler temperature set to 10°C, and column temperature set to 30°C.
Mobile Phase A consisted of 20 mM ammonium carbonate and 0.1\% (v/v) ammonium hydroxide, and Mobile Phase B consisted of 100\% acetonitrile.
The sample was gradient eluted (\%B) from the column as follows: 0-20 min.: linear gradient from 85\% to 20\% B; 20-24 min.: hold at 20\% B; 24-24.5 min.: linear gradient from 20\% to 85\% B; 24.5 min.-end: hold at 85\% B until equilibrated with ten column volumes.
Mobile Phase was directed into the ion source with the following parameters: sheath gas = 45, auxiliary gas = 15, sweep gas = 2, spray voltage = 2.9 kV in the negative mode or 3.5 kV in the positive mode, capillary temperature = 300°C, RF level = 40\%, auxiliary gas heater temperature = 325°C.
Mass detection was conducted with a resolution of 240,000 in full scan mode, with an AGC target of 3,000,000 and maximum injection time of 250 msec.
Metabolites were detected over a mass range of 70-850 m/z.
Quantitation of all metabolites was performed using Tracefinder 4.1 (Thermo Scientific) referencing an in-house metabolite standards library using ≤5 ppm mass error.
For samples subjected to stable isotope tracing, peak areas were natural abundance corrected with IsoCor \cite{Millard2019-hv}, using experimentally determined tracer purity values.



\subsection{Media uptake flux}
The cells were first passaged in DMEM with dialyzed FBS and the tracer and/or metabolites used during the uptake experiment.
For 143B GOT DKO cells expressing SLC1A3, 500 µM sodium aspartate was added to adjust intracellular aspartate levels.
For 143B WT cells, 100 µM U-\hCi{} Asn was added to achieve a steady-state label fraction in the proteome.
To start the experiment, cells were seeded on six well dishes at 1e5 cells/well.
On the next day fresh media was added and t=0 media samples were collected.
Then cells were incubated and subsequent media samples collected.
After the last media collection the residual media volume was quantified to correct for evaporation
For U-\hCi{} Asn tracing the labelling ratio was determined by extracting intracellular metabolites after the last media collection.
Two dishes were run in parallel and used for counting to determine proliferation rates and cell volume measurement using a coulter counter.

\subsection{Acid hydrolysis to measure amino acids and pyrimidines}
Two 12W plates seeded in parallel (DMEM dia. FBS)
At 40-90\% confluency harvest by washing four times with saline
Count one plate
On the other plate add 1 mL 6M HCl to 5 wells, seal the plate and perform acid hydrolysis at 90C in incubator for 20h
Move hydrolysate to 2 mL tubes then wash the well thrice with 1 mL water and collect it in the same tube
Dry the hydrolysate
Add 0.5 mL 6M HCl to each tube and incubate another 48h at 90C, then dry
Reconstitute in 0.5 mL 6M HCl and aliquot 2 tubes with 10k cells and 2 tubes with 40k cells and dry.
Reconstitute the two 10k tubes with 1 mL water, move to a fresh tube, dry and reconstitute these with 40 uL SE+CAAv2 and run standard LCMS
Water reconstitution is an attempt to get rid of the plastic/glue soluble in HCl
Using uracil as a control (should be in the same range regardless of scraping or on plate hydrolysis)
% Sigma	84429-10X2ML	HCl for amino acid analysis




\subsubsection{Flux calculation}
Cell counts over time is assumed an exponential function ($y(t)$) with proliferation rate $K$ and cell count at $t=0$ being $y_0$:
\begin{equation}
    y(t) = y_0 2^{K t}
\end{equation}

Assuming amino acid uptake into a cell is constant, the uptake rate (also called influx) can be defined as the total molar uptake over a time range divided by the total area under the cell count curve of the same time range.
Formally, the flux of a compound $i$ is:
\begin{equation}
    F_i = \frac{n_i(t_1) - n_i(t_2)}{\int_{t_1}^{t_2} y(t) dt}
\end{equation}
With $F_i$ being the flux, $n_i(t)$ being the molar quantity of a compound $i$, and $t_1$ and $t_2$ being the first and second timepoint, respectively.

Using $\Delta$ to indicate the difference between $t_1$ and $t_2$, we can simplify the above:
\begin{equation}
    F_i = \frac{\Delta n_i}{\int_{t_1}^{t_2} y(t) dt} = \frac{\Delta n_i}{\frac{y_0 2^{K t_2}}{K\ \natlog(2)} - \frac{y_0 2^{K t_1}}{K\ \natlog(2)}} = \frac{\Delta n_i}{\frac{y_0 (2^{K t_2} - 2^{K t_1})}{K\ \natlog(2)}}
\end{equation}
The denominator is the area under the cell count curve from $t_1$ to $t_2$ and is sometimes referred to as ``cell hours'' because of its unit is time, typically hours.

Using cell hours it is possible to calculate the molar quantity taken up per cell per hour, typically as: $\frac{\text{fmol}}{\text{cell}\times h}$.
However, this makes it hard to compare across cell lines because of variability in cell size.
To fix this, simply redefine the problem from integrating the area under the cell count curve to the area under the cell volume curve.
The cell volume curve is defined using a volume per cell multiplier ($V_c$):
\begin{equation}
    V(t) = y(t) V_c = V_c y_0 2^{t K}
\end{equation}

Assuming cell volume is unchanged throughout the experiment, $V_c$ is a constant that is not integrated and we get:
\begin{equation}
    F_i = \frac{\Delta n_i}{\frac{V_c y_0 (2^{K t_2} - 2^{K t_1})}{K\ \natlog(2)}}
\label{eq:ch2:fl_vh}
\end{equation}
Since $V_c$ is a volume, we have a denominator with a typical unit of $\frac{L}{h}$.
We could call this ``volume hours'', and dividing a molar quantity onto this we typical get a unit of $\frac{\text{mM}}{h}$, which is comparable across cell lines with different sizes.

Sometimes it can be useful to convert the influx of a compound to its accumulated intracellular concentration, also referred to as total cell concentration.
The accumulated intracellular concentration of a compound can be understood intuitively as the total molar uptake divided by the total increase in cell volume.
To convert flux to total cell concentration for compound $i$ ($C_i$) observe the following rearrangement of equation \ref{eq:ch2:fl_vh}:
\begin{equation}
    F_i = K\ \natlog(2) \frac{\Delta n_i}{V_c y_0 (2^{K t_2} - 2^{K t_1})}
\end{equation}
The fraction is the total cell concentration for compound $i$ and thus we have:
\begin{equation}
    F_i = K\ \natlog(2) C_i => C_i = \frac{F_i}{K\ \natlog(2)}
\end{equation}


\subsubsection{Net asparagine consumption}
Human cells do not have any appreciable asparagine deaminase activity \cite{Sullivan2018-gz} and thus asparagine is a terminal metabolite that does not get converted or used as a substrate in other metabolic reactions.
This makes the asparagine fluxes suitable for isotope tracing if we make the explicit assumption is that asparagine can be generated from synthesis using aspartate but not recycled back.
Figure \ref{fig:ch2:asn_Jprot} shows the net asparagine fluxes when U-\hCi{} labelled Asn is added to the media.
Each net flux is the sum of fluxes in both directions e.g. U-\hCi{} Asn is net influxed because it is consumed while unlabelled Asn is net effluxed because it is absent from media at the initial conditions.

Influx and efflux can be measured by media sampling and using the ratio of labelled to unlabelled Asn inside the cell, the remaining fluxes can be solved.
Observe that the labelling ratio is defined by the influx, efflux and synthesis flux:
\begin{equation}
    \frac{\UAsn}{\Asn} = \frac{\Flin}{\Flsyn - \Flout}
\end{equation}

Isolate synthesis flux:
\begin{equation}
    \Flsyn = \frac{\Asn}{\UAsn} \Flin + \Flout
\label{eq:ch2:Jsyn}
\end{equation}

Introduce flux balance:
\begin{equation}
    \Flin + \Flsyn = \Flprot + \Flout => \Flprot = \Flin + (\Flsyn - \Flout)
\end{equation}

Isolate the flux of Asn deposition into protein ($\Flprot$) and insert $\Flsyn$ from equation \ref{eq:ch2:Jsyn}:
\begin{equation}
    \Flprot = \Flin + \left( \frac{\Asn}{\UAsn} \Flin + \Flout - \Flout \right)
\end{equation}

Simplify:
\begin{equation}
    \Flprot = \Flin \left( 1 + \frac{\Asn}{\UAsn} \right)
\label{eq:ch2:Jprot}
\end{equation}



\subsubsection{Aspartate consumption flux}
% How was it calculated
%%% purines = pyrimidines plus a 2 fold correction for adenine
% Table with the raw data values











\subsection{Nitrogen-15 tracing}

\subsubsection{Salvage fraction of individual components}
The fractional contribution of individual components into their respective aspartate consuming fates was determined in 143B and H1299 cells for the salvageable metabolites asparagine (Asn), uridine (Urd), hypoxanthine (Hpx), adenine (Ade) and guanine (Gua) along with a vehicle treatment (Vec).
The salvageable metabolites were spiked-in from a 20x stock solution to achieve a final concentration of: 500 µM Asn, 200 µM Urd, 100 µM Hpx, 100 µM Ade or 100 µM Gua.
The fraction of salvage was determined by stable isotope tracing, performed using both Gln amide\=/\hNi{} (Cambridge Isotope Laboratories, NLM-557-PK) and Gln alpha\=/\hNi{} (Cambridge Isotope Laboratories, NLM-1016-PK) in separate reactions and added to DMEM without glucose, glutamine, pyruvate and phenol red (Sigma, D5030) supplemented with 10\% dialyzed FBS, 1\% penicillin-streptomycin, 25 mM glucose (Sigma, G7528).
The combination of cell lines, salvageable metabolites and tracers gave 2x6x2=24 conditions which were labelled to steady-state by culturing for four passages with a 1/20 split at each passage.
At the end of the last passage each condition was split into four technical replicates and plated on 24 well dishes (Corning, 3524).
Upon reaching confluency, polar metabolites were extracted and submitted to LCMS with the above described technique.

\subsubsection{Salvage fraction as a function of concentration}
Seeded H1299 and 143B cells at 5,000 and 10,000 cells/well on a 6-well dish in DMEM containing Gln amide\=/\hNi{} as described above.
Then added an equimolar mix of hypoxanthine and uridine from a 40x stock to a final concentration of 10, 20, 30, 40, 50, 60 µM in each of the 6 wells.
Fresh media was then added every four days and upon reaching confluency, polar metabolites were extracted and submitted to LCMS with the above described technique.






\subsection{ATF4 reporter measurements}

% GCN2i concentration



\subsection{Plotting and statistics}
Plots with matplotlib and seaborn.
Errorbars are bootstrapped 95\% CI and +/- stdev








 % Molecular mechanisms of aspartate limitation

% Asp-sensor manuscript:
\chapter{An engineered biosensor enables dynamic aspartate measurements in living cells}
blaa

\section{Introduction}
More blaa % An engineered biosensor enables dynamic aspartate measurements in living cells

% Describe all things relevant to understand tRNA-Seq method:
%KD: Hell! Too lazy to write this crap.
%\chapter{First degree dependence of aspartate for cell proliferation}

Closing chapter, dumping all the shitty unpublished data here.

\section{Introduction}
More blaa % Introduction to tRNA biology

% tRNA-Seq manuscript:
\chapter{Robust method for measuring aminoacylation through tRNA-seq}
Insert tRNA-Seq manuscript here.
 % Robust method for measuring aminoacylation through tRNA-seq

% Future directions:
\chapter{Future directions}

Closing chapter 

\section{Introduction}
More blaa

Cell size the nucleotide imbalance hypothesis.
%%% See "ETCrescue" folder under lab-work



\begin{figure}
    \centering
    \includegraphics[width=0.85\textwidth]{figures/chap6/P_Chk_wstrn.pdf}
    \caption[gggg]{
    gggg
    }
    \label{fig:ch6:P_Chk_wstrn}
\end{figure}









%
% ==========   Bibliography
%
% \nocite{*}   % include everything in the uwthesis.bib file
\bibliographystyle{plain}
\bibliography{uwthesis}
%
% ==========   Appendices
%
\appendix
\raggedbottom\sloppy
 
% ========== Appendix A


\chapter{Mathematical description of simple labelling flux}
\label{chap1_app}
A common goal of stable isotope tracing is the measurement of flux through a pathway.
This requires a mathematical description of the system with a set of associated assumptions.
Here such description is provided for a very simple system describing a single metabolic reaction.

The system is illustrated in figure \ref{fig:app_ch1:tracing_diagram} and contains one or more cells with total volume $V(t)$, proliferating with growth rate $g$ and two metabolites with the concentrations $A$ and $B$.
$A_{out}$ is situated outside of the cell, upon transport into the cell at a rate $r_{in}$, it becomes available for conversion into $B$ at a rate $r_{AB}$, whereafter it is exported outside the cell at a rate $r_{out}$.
These metabolites are either labelled ($A^L$) or unlabelled ($A^U$) and sum describes the total concentration i.e. $A = A^L + A^U$.

For this system a number of assumptions are made:
1) cell proliferation is constant,
2) the concentrations $A_{out}$, $A$ and $B$ are constant,
3) all reactions (indicated by arrows) are irreversible,
4) the labelling fraction of $A$ is constant and the same both inside and outside.

The objective is to determine the flux from $A$ to $B$ ($F_{AB} = A\ r_{AB}$) in order to describe $B^L$ and $B^U$ over time.

\begin{figure}[ht]
    \centering
    \includegraphics[width=0.7\textwidth]{figures/chap1/app/tracing_diagram.pdf}
    \caption[Mass balance diagram for metabolite labelling]{
    Mass balance diagram for labelling of metabolite $B$.
    See text for details.
    }
    \label{fig:app_ch1:tracing_diagram}
\end{figure}


To deal with labelled and unlabelled metabolites the labelling fraction, which is constant for $A$, is introduced:
\begin{equation}
    \alpha = \frac{A^L}{A^L +A^U} = \frac{A^L}{A} \Longrightarrow A^L = \alpha A
\label{eq:app_ch1:alpha}
\end{equation}

For $B$ the labelling fraction changes over time:
\begin{equation}
    \beta(t) = \frac{B^L}{B^L +B^U} = \frac{B^L}{B} \Longrightarrow B^L = \beta B
\label{eq:app_ch1:beta}
\end{equation}

For the cell volume we know that it follows standard exponential growth with initial condition $V(0) = V_0$ such that:
\begin{equation}
    V(t) = V_0 e^{g t}
\label{eq:app_ch1:cell_vol}
\end{equation}
This could also be derived from the mass balance, assuming constant cell density.

Now, applying mass balance of the control volume (IN - OUT = ACCUMULATED) on $A^L$ over a time difference of $\Delta t$:
\begin{equation}
    V \alpha A_{out}\ r_{in} \Delta t - V \alpha A\ r_{AB} \Delta t = (V \alpha A)(t + \Delta t) - (V \alpha A)(t)
\label{eq:app_ch1:A_balance}
\end{equation}
Rearrange:
\begin{equation}
    \frac{(V \alpha A)(t + \Delta t) - (V \alpha A)(t)}{\Delta t} = V \alpha A_{out}\ r_{in} - V \alpha A\ r_{AB}
\label{eq:app_ch1:A_dq1}
\end{equation}
Let $\Delta t \rightarrow 0$ and the differential quotient emerge:
\begin{equation}
    \frac{d}{dt} (V \alpha A) = V \alpha A_{out}\ r_{in} - V \alpha A\ r_{AB} = \alpha A \frac{d V}{dt} + V \alpha \frac{dA}{dt}
\label{eq:app_ch1:A_dq2}
\end{equation}
From assumption 4) above it is known that $\frac{dA}{dt} = 0$, also from the volume equation (equation \ref{eq:app_ch1:cell_vol}) it is observed that $\frac{dV}{dt} = V g$:
\begin{equation}
    \alpha A V\ g = V \alpha A_{out}\ r_{in} - V \alpha A\ r_{AB} \Longrightarrow A_{out}\ r_{in} = A\ r_{AB} + A\ g
\label{eq:app_ch1:A_dq2s}
\end{equation}
This can be interpreted as influx ($F_{in}$) is equal to consumption flux to B ($F_{AB}$) plus flux to maintain constant concentration $A$ while cells are replicating ($F^A_{rep}$):
\begin{equation}
    A_{out}\ r_{in} = A\ r_{AB} + A\ g \Longrightarrow F_{in} = F_{AB} + F^A_{rep}
\label{eq:app_ch1:A_flux}
\end{equation}


Similarly, applying mass balance on $B^L$:
\begin{equation}
    V \alpha A\ r_{AB} \Delta t - V \beta B\ r_{out} \Delta t = (V \beta B)(t + \Delta t) - (V \beta B)(t)
\label{eq:app_ch1:B_balance}
\end{equation}
\begin{equation}
    \frac{(V \beta B)(t + \Delta t) - (V \beta B)(t)}{\Delta t} = V \alpha A\ r_{AB} - V \beta B\ r_{out}
\label{eq:app_ch1:B_dq1}
\end{equation}
\begin{equation}
    \frac{d}{dt} (V \beta B) = V \alpha A\ r_{AB} - V \beta B\ r_{out} = \beta B \frac{d V}{dt} + V B \frac{d\beta}{dt}
\label{eq:app_ch1:B_dq2}
\end{equation}
\begin{equation}
    \beta B\ g + B \frac{d\beta}{dt} = \alpha A\ r_{AB} - \beta B\ r_{out}
\label{eq:app_ch1:B_dq2s}
\end{equation}
The flux from $A$ to $B$  has already been defined as $F_{AB} = A\ r_{AB}$, also let $F_{out} = B\ r_{out}$ and $F^B_{rep} = B\ g$:
\begin{equation}
     \beta F^B_{rep} + B \frac{d\beta}{dt} = \alpha F_{AB} - \beta F_{out}
\label{eq:app_ch1:B_dq2s2}
\end{equation}
Rearrange:
\begin{equation}
     B \frac{d\beta}{dt} = \alpha F_{AB} - \beta (F_{out} - F^B_{rep})
\label{eq:app_ch1:B_dq2s3}
\end{equation}
Observe, similar to equation \ref{eq:app_ch1:A_flux} that $F_{out} = F_{AB} + F^B_{rep}$ and insert:
\begin{equation}
     \frac{d\beta}{dt} = (\alpha - \beta) \frac{F_{AB}}{B}
\label{eq:app_ch1:B_dq2s4}
\end{equation}
By integration the labelling fraction of $B$ is found:
\begin{equation}
     \beta(t) = \alpha + C e^{-\frac{F_{AB}}{B} t}
\label{eq:app_ch1:B_t}
\end{equation}
In most cases there will be no labelling of $B$ at time zero i.e. an initial condition of $\beta(t) = 0$, resulting in integration constant $C = -\alpha$:
\begin{equation}
     \beta(t) = \alpha - \alpha e^{-\frac{F_{AB}}{B} t}
\label{eq:app_ch1:B_t_IC}
\end{equation}


A useful descriptor is the time to half-max labelling, or more generally the time ($t_{\lambda}$) to fraction-of-max ($\lambda \in [0, 1]$) labelling:
\begin{equation}
     \lambda = \frac{\beta(t_{\lambda})}{\beta(\infty)} = \frac{\alpha - \alpha e^{-\frac{F_{AB}}{B} t_{\lambda}}}{\alpha}
\label{eq:app_ch1:B_hl}
\end{equation}
Isolating the time:
\begin{equation}
     t_{\lambda} = \frac{B}{F_{AB}} \ln\left( \frac{1}{1 - \lambda} \right)
\label{eq:app_ch1:B_hl2}
\end{equation}

Finally as an example, glutamine to glutamate synthesis flux is estimate to 30 mM/h and the intracellular pool of glutamate is estimate to a concentration of 5 mM.
How long does it take to reach 95\% of max labelling from glutamine?
\begin{equation}
     \frac{5\ mM}{30\ mM/h} \ln\left( \frac{1}{1 - 0.95} \right) \approx 0.5\ h
\label{eq:app_ch1:B_glu}
\end{equation}






\chapter{Nucleoside and nucleobase quantification}



\begin{figure}
    \centering
    \includegraphics[width=0.45\textwidth]{figures/chap2/arg_syn.pdf}
    \caption[No evidence of arginine synthesis in DMEM]{
    No evidence of arginine synthesis in DMEM.
    Top diagram shows Gln alpha\=/\hNi{} label incorporation into Asp and subsequently Arg.
    Bottom isotopologue distribution shows Gln alpha\=/\hNi{} label incorporation into Gln, Asp and Arg at steady-state for cell lines 143B and H1299 grown in DMEM.
    }
    \label{fig:app_ch2:arg_syn}
\end{figure}










\begin{figure}
    \centering
    \includegraphics[width=0.95\textwidth]{figures/chap2/app/atf4_chVSsp.pdf}
    \caption[APP GGGG]{
    gggg
    }
    \label{fig:app_ch2:sal_frac_conc}
\end{figure}


\begin{figure}
    \centering
    \includegraphics[width=0.95\textwidth]{figures/chap2/app/HT1080_ISR_western.pdf}
    \caption[APP GGGG]{
    gggg
    }
    \label{fig:app_ch2:HT1080_ISR_western}
\end{figure}



\begin{figure}
    \centering
    \includegraphics[width=0.95\textwidth]{figures/chap2/app/atf4_ETCtit.pdf}
    \caption[APP GGGG]{
    gggg
    }
    \label{fig:app_ch2:atf4_ETCtit}
\end{figure}




\begin{figure}
     \centering
     \begin{subfigure}[b]{0.49\textwidth}
         \includegraphics[width=\textwidth]{figures/chap2/app/HT1080_ETCinhib_ATF4rep_low.pdf}
         \caption{ggg}
         \label{fig:app_ch2:HT1080_ETCinhib_ATF4rep_low}
     \end{subfigure}
     \hfill
     \begin{subfigure}[b]{0.49\textwidth}
         \includegraphics[width=\textwidth]{figures/chap2/143B_ETCinhib_ATF4rep_high.pdf}
         \caption{ggg}
         \label{fig:app_ch2:143B_ETCinhib_ATF4rep_high}
     \end{subfigure}
     \hfill
     \begin{subfigure}[b]{0.4\textwidth}
         \includegraphics[width=\textwidth]{figures/chap2/HT1080_Atp_ATF4rep.pdf}
         \caption{ggg}
         \label{fig:app_ch2:HT1080_Atp_ATF4rep}
     \end{subfigure}
        \caption[ggg]{
        gggg
        }
        \label{fig:app_ch2:ISR}
\end{figure}












\chapter{Aspartate sensor supplementary figures}

\begin{figure}[ht!]
    \centering
    \fbox{\includegraphics[width=0.7\linewidth]{figures/chap3/Fig1S1.pdf}}
    \caption[Aspartate specificity and excitation/emission spectra.]{
    (A) Switching specificity of the iGluSnFR3 precursor from glutamate to aspartate using S72X library (left) and S72P, S27X library (right).
    Titrations with aspartate (solid lines) and glutamate (dashed lines) in bacterial lysate.
    (B) Excitation and emission spectra of jAspSnFR3-mRuby3.
    Left, 1-photon spectra.
    Excitation wavelength was varied from 400 nm to 520 nm (7.5 nm bandpass) while observing emission at 535 nm (10 nm bandpass).
    Emission wavelength was varied from 535 nm to 600 nm (10 nm bandpass) while exciting at 510 nm (7.5 nm bandpass).
    Fluorescence was measured both in the absence (dashed lines) and presence of 10 mM aspartate (solid lines).
    Right, 2-photon cross-sections, also ± 10 mM aspartate, with an overlay of calculated $\Delta$F/F (green).
    Vertical bar indicates 1040 nm.
    }
    \label{ch3:figsupp:f1S1}
\end{figure}

\begin{figure}[ht]
    \centering
    \fbox{\includegraphics[width=0.7\linewidth]{figures/chap3/Fig1S2.pdf}}
    \caption[Decoy, temperature and pH sensitivity.]{
    (A) jAspSnFR3-mRuby3 does not appreciably change its green fluorescence in response to other amino acids (alanine, phenylalanine, glycine, histidine (red line), isoleucine, leucine, methionine, proline, glutamine, arginine, serine, threonine, valine, or tryptophan).
    Insert with aspartate in black and glutamate/asparagine in grey for comparison.
    Ex. 485 nm (20 nm bandpass), Em. 535 nm (20 nm bandpass), 0.2 $\mu$M purified protein in PBS.
    (B) jAspSnFR3-mRuby3 does not respond to other decoys: citrate, lactate, pyruvate, malate, alpha-ketoglutarate, cis-aconitate, succinate, fumarate, or oxaloacetate (orange squares); nor to relevant pharmacological treatments: rotenone (green squares) or metformin.
    The small increase in fluorescence from rotenone is likely due to the scattering of a visibly turbid solution; rotenone has very low solubility in water.
    Ex. 485 nm (20 nm bandpass), Em. 535 nm (20 nm bandpass).
    (C) jAspSnFR3-mRuby3 is not adversely affected by temperature.
    Fluorescence as a function of aspartate titration at 23°C (light grey), 30°C (medium grey), and 37°C (black).
    Error bars are standard deviation of three technical replicates.
    (D) pH sensitivity of jAspSnFR3-mRuby3 (green component).
    Ex 485 nm (5 nm bp), Em 515 nm (10 nm bp).
    Error bars are standard deviation of 5 technical replicates.
    Solid line is with 3 mM aspartate, dashed line is without aspartate.
    }
    \label{ch3:figsupp:f1S2}
\end{figure}

\begin{figure}[ht]
    \centering
    \fbox{\includegraphics[width=0.8\linewidth]{figures/chap3/Fig1S3.pdf}}
    \caption[Stopped-flow kinetics.]{
    (A) Stopped-flow kinetics of jAspSnFR3 using different aspartate concentrations for injection.
    Measurements were performed at a frequency of 1 per msec. and indicated by dots.
    To each time-series an exponential function was fit, shown as a solid line with matching color.
    (B) $k_{obs}$ as a function of aspartate concentration shown for two independent stopped-flow experiments.
    The line represents the linear function, fitted to the linear range to extract the kinetic rates.
    }
    \label{ch3:figsupp:f1S3}
\end{figure}

\begin{figure}[ht]
    \centering
    \fbox{\includegraphics[width=0.7\linewidth]{figures/chap3/Fig1S4.pdf}}
    \caption[mRuby3 interaction with histidine tag.]{
    (A) jAspSnFR3-mRuby3 shows increased red fluorescence at millimolar concentrations of all amino acids amino acids (alanine, phenylalanine, glycine, histidine, isoleucine, leucine, methionine, proline, glutamine, arginine, serine, threonine, valine, or tryptophan), with apparent responses to histidine at 100 $\mu$M (red line).
    (B) Increased red fluorescence of mRuby3 in response to histidine requires a C-terminal histidine tag.
    (C) Increased red fluorescence of mRuby3 in response to millimolar concentrations of aspartate requires a C-terminal histidine tag.
    (D) mRuby, with or without histidine tag, does not increase in fluorescence upon treatment with amino acid related compound gamma-aminobutyric acid (GABA).
    For all plots Ex. 555 nm (20 nm bandpass), Em. 600 nm (20 nm bandpass).
    }
    \label{ch3:figsupp:f1S4}
\end{figure}

\begin{figure}[ht]
    \centering
    \fbox{\includegraphics[width=0.98\linewidth]{figures/chap3/Fig2S1.pdf}}
    \caption[Rotenone titration in different cell lines.]{
    jAspSnFR3 temporal response after rotenone treatment.
    (A) HT1080 cells using nuclear RFP to normalize the jAspSnFR3 signal, treated with a rotenone titration.
    (B) HT1080 cells using an RFP fused to jAspSnFR3 (jAspSnFR3-mRuby3) for normalization, treated with a rotenone titration.
    (C) HT1080 cells treated with 100 nM rotenone at the start of the experiment (0 h) and then rescued with pyruvate at start, 22 h or never.
    (D) Comparison between the steady-state signal of (A) and (B) with a linear regression shown as a red dashed line to show that nuclear RFP and RFP fusion normalizations are equivalent.
    (E) HEK293t cells using an RFP fused to jAspSnFR3 for normalization, treated with a rotenone titration.
    For plots (A), (B), (C) and (E) markers indicate the average using available well replicates and are superimposed on a bootstrapped 95\% confidence interval colored using the same color code as the markers.
    For plot (D) markers indicate the average using available well replicates and errorbars are drawn as +/- the standard deviation of the replicates.
    Grey dashed lines indicate the time of treatment.
    Orange dashed line in panel (C) indicates time of pyruvate addition.
    AU, arbitrary unit.
    }
    \label{ch3:figsupp:f2S1}
\end{figure}

\begin{figure}[ht]
    \centering
    \fbox{\includegraphics[width=0.9\linewidth]{figures/chap3/Fig2S2.pdf}}
    \caption[Plots related to glutamine limitation.]{
    (A) Nuclei count over time for conditions displayed in figure \ref{ch3:fig:Fig2}, panel E.
    (B) Cell confluency over time for conditions displayed in figure \ref{ch3:fig:Fig2}, panel E.
    (C) H1299 cells changed into media with a titration of glutamine with or without 1 mM asparagine.
    Identical to figure \ref{ch3:fig:Fig2}, panel H but with fewer glutamine concentrations and more well replicates.
    }
    \label{ch3:figsupp:f2S2}
\end{figure}

\begin{figure}[ht]
    \centering
    \fbox{\includegraphics[width=0.98\linewidth]{figures/chap3/Fig3S1.pdf}}
    \caption[jAspSnFR3 signal does not correlate with glutamate concentration.]{
    RFP normalized jAspSnFR3 signal, following various perturbations to live cells, is not correlated with the LCMS measured intracellular glutamate concentration.
    Datapoints are fitted to a local linear regression, shown by the black line, otherwise, these plots are identical to those in figure \ref{ch3:fig:Fig3}.
    AU, arbitrary unit.
    }
    \label{ch3:figsupp:f3S1}
\end{figure}







\chapter{tRNA-Seq supplementary figures}

cccccc



% \chapter{Secret appendix}
You are now reading the secret appendix!
A place hosting information classified due to reasons such as: high variance, unreliable controls, missing validations, partial knockouts, swapped labels, performed by undergrads, irregular proliferation, high background, bad antibodies, pipetting accidents, wrong volumes, low signal or just because it does not fit anywhere else.



\section{Integrated stress response}

\subsection{Aspartate depletion induced ISR}
Aspartate depletion is achieved in GOT DKO cells and shown to cause integrated stress response (ISR) both through eIF2a phosphorylation and ATF4 upregulation.
Media swapping induces a quick depletion of both Asp and Asn (figure \ref{fig:sapp:ISR:143B_GOT_DKO_ISR_conc}).
Asn efflux is likely more quick due to better permeability.
Upon Asn depletion the ISR cascade starts and can maintain high expression of ATF4 due to the continued Asn synthesis from Asp.
On the other hand, if Asp is depleted to the point of protein synthesis inhibition no signal might be detected when probing ATF4 as Asp levels will not recover.
We observe these dynamics clearly in HT1080 GOT DKO cells in figures \ref{fig:sapp:ISR:HT1080_DKO_ISR} and \ref{fig:sapp:ISR:HT1080_DKO_ASPtit_time}.
Especially, the ATF4 reporter shows the kinetics of these Asp/Asn depletion kinetics.

For 143B GOT DKO we observe similar results.
Here, inhibition of GCN2 ablates ATF4 expression while mitochondrial respiration (missing in rho0 cells) is not required for ATF4 upregulation.

\begin{figure}[t]
    \centering
    \includegraphics[width=0.99\textwidth]{figures/sapp/ISR/HT1080_DKO_ISR.pdf}
    \caption[Asp depl. induced ISR, HT1080 western]{
    Aspartate depletion in HT1080 GOT DKO cells initiated by media swapping.
    }
    \label{fig:sapp:ISR:HT1080_DKO_ISR}
\end{figure}

\begin{figure}[t]
    \centering
    \includegraphics[width=0.98\textwidth]{figures/sapp/ISR/HT1080_DKO_ASPtit_time.pdf}
    \caption[Asp depl. induced ISR, HT1080 ATF4 reporter]{
    ATF4 reporter measurements after aspartate depletion in HT1080 GOT DKO (clone with low reporter at baseline).
    Vec/Vec normalization is normalization to the baseline condition (20 mM Asp, no Asn).
    Vertical red line on curve plot indicates time of measurements extracted for the barplot.
    }
    \label{fig:sapp:ISR:HT1080_DKO_ASPtit_time}
\end{figure}

\begin{figure}[t]
    \centering
    \includegraphics[width=0.60\textwidth]{figures/sapp/ISR/143B_GCN2i_val.pdf}
    \caption[GCN2 inhibitor validation]{
    Validation of activity and specificity of GCN2i in 143B WT cells.
    }
    \label{fig:sapp:ISR:143B_GCN2i_val}
\end{figure}

\begin{figure}[t]
    \centering
    \includegraphics[height=0.85\textheight]{figures/sapp/ISR/143B_DKO_ISR.pdf}
    \caption[Asp depl. induced ISR, 143B western]{
    Aspartate depletion in 143B GOT DKO, SLC1A3 cells initiated by media swapping.
    }
    \label{fig:sapp:ISR:143B_DKO_ISR}
\end{figure}

\begin{figure}[!ht]
     \centering
     \begin{subfigure}[b]{0.35\textwidth}
         \includegraphics[width=\textwidth]{figures/sapp/ISR/143B_GOT_DKO_ISR_Asp_conc.pdf}
         \caption{Intracellular Asp}
         \label{fig:sapp:ISR:143B_GOT_DKO_ISR_Asp_conc}
     \end{subfigure}
     \hspace{0.02\textwidth}
     \begin{subfigure}[b]{0.35\textwidth}
         \includegraphics[width=\textwidth]{figures/sapp/ISR/143B_GOT_DKO_ISR_Asn_conc.pdf}
         \caption{Intracellular Asn}
         \label{fig:sapp:ISR:143B_GOT_DKO_ISR_Asn_conc}
     \end{subfigure}
     \hfill
        \caption[Intracellular Asp/Asn at ISR in GOT DKO]{
        Intracellular concentration of aspartate and asparagine before and 2 hours after media switch with/without aspartate for 143B GOT DKO cells, similar to figure \ref{fig:sapp:ISR:143B_DKO_ISR} (second panel from the top).
        }
        \label{fig:sapp:ISR:143B_GOT_DKO_ISR_conc}
\end{figure}





\FloatBarrier
\subsection{OMA1/HRI relation to rotenone/antimycin induced ISR}
According Fessler et al. and Guo et al. \cite{Fessler2020-zk, Guo2020-ia} OMA1/HRI is required for FCCP induced ISR.
Guo et al. also shows OMA1/HRI is required for rotenone/antimycin induced ISR (suppl. of Guo et al.).

We have pooled knockout cells (parental HT1080 ATF4 reporter low clone) of OMA1 and HRI.
These appear to ablate rotenone/antimycin induced ISR but strangely not FCCP induced ISR.
This could be due to pleiotropic effect of FCCP e.g. its is also depolarizing the plasma membrane and the lysosomal membrane.

\begin{figure}[!ht]
     \centering
     \begin{subfigure}[b]{0.49\textwidth}
         \includegraphics[width=\textwidth]{figures/sapp/ISR/ATF4rep_OMA1pool.pdf}
         \caption{OMA1 KO pool}
         \label{fig:sapp:ISR:ATF4rep_OMA1pool}
     \end{subfigure}
     \hfill
     \begin{subfigure}[b]{0.49\textwidth}
         \includegraphics[width=\textwidth]{figures/sapp/ISR/ATF4rep_HRIpool.pdf}
         \caption{HRI KO pool}
         \label{fig:sapp:ISR:ATF4rep_HRIpool}
     \end{subfigure}
     \hfill
        \caption[ATF4 post mito inhib. OMA1/HRI KO, reporter]{
        ATF4 reporter assay measured 21 h after drug treatment with vehicle, rotenone (100 nM) or antimycin (1 µM) spiked-in as 10x.
        }
        \label{fig:sapp:ISR:ATF4rep_OMA1_HRIpool}
\end{figure}

\begin{figure}[t]
    \centering
    \includegraphics[width=0.65\textwidth]{figures/sapp/ISR/ATF4wes_OMA1_HRIpool.pdf}
    \caption[ATF4 post mito inhib. OMA1/HRI KO, western]{
    Same pooled knockout cells and handling as in figure \ref{fig:sapp:ISR:ATF4rep_OMA1_HRIpool}.
    Drugs spiked-in as 10x.
    }
    \label{fig:sapp:ISR:ATF4wes_OMA1_HRIpool}
\end{figure}






\FloatBarrier
\subsection{OMA1/HRI relation to asp depletion induced ISR}
OMA1 KO appears on western blot to ablate ATF4 upregulation in response to aspartate depletion; however, HRI does not.
Another experiment, exclusively with OMA1, shows that it is not required for ATF4 upregulation after aspartate depletion, on the other hand GCN2 appears to be required.

If OMA1/HRI is involved with aspartate depletion induced ISR it could be through altering the mitochondrial membrane potential.
Aspartate depletion would prevent the malate-aspartate from shuttling electrons into the inner mitochondria, a process which is, presumably, still active in GOT DKO cells.
Aspartate is exported out of the mitochondria using the membrane potential i.e. one aspartate out for one glutamate and a proton in.
Thus, depleting cytoplasmic aspartate might alter the membrane potential in unpredictable ways and indeed there appears to be a small increase in mitochondrial membrane potential in HT1080 GOT DKO cells as aspartate is progressively depleted from the media (figure \ref{fig:sapp:ISR:HT1080_GOT_DKO_TMRE}).

\begin{figure}[ht]
    \centering
    \includegraphics[width=0.95\textwidth]{figures/sapp/ISR/HT1080_DKO_KO_ISR.pdf}
    \caption[ATF4 post Asp depl. OMA1/HRI KO, western]{
    Effect of OMA1, HRI or DARS2 KO on aspartate depletion induced ISR.
    Single cell clones validated for OMA1 and DARS2, only functionally validated for HRI, see figure \ref{fig:sapp:ISR:OMA1_HRI_DARS2_val}.
    Aspartate depletion initiated by switching to media with no asp at time zero.
    Irrelevant wells spliced out.
    }
    \label{fig:sapp:ISR:HT1080_DKO_KO_ISR}
\end{figure}

\begin{figure}[ht]
     \centering
     \begin{subfigure}[b]{0.49\textwidth}
         \includegraphics[width=\textwidth]{figures/sapp/ISR/HT1080_GOT_DKO_HRI_KO.jpeg}
         \caption{HT1080 GOT DKO, HRI KO}
         \label{fig:sapp:ISR:HT1080_GOT_DKO_HRI_KO}
     \end{subfigure}
     \hfill
     \begin{subfigure}[b]{0.49\textwidth}
         \includegraphics[width=\textwidth]{figures/sapp/ISR/HT1080_GOT_DKO_OMA1_DARS2_KO.jpeg}
         \caption{HT1080 GOT DKO, OMA1/DARS2 KO}
         \label{fig:sapp:ISR:HT1080_GOT_DKO_OMA1_DARS2_KO}
     \end{subfigure}
     \hfill
        \caption[HT1080 GOT DKO, OMA1/HRI/DARS2 KO validation]{
        Western blot validations of OMA1, HRI and DARS2 knockouts.
        Image art credit: Ian Engstrom.
        }
        \label{fig:sapp:ISR:OMA1_HRI_DARS2_val}
\end{figure}

\begin{figure}[!ht]
     \centering
     \begin{subfigure}[b]{0.6\textwidth}
         \includegraphics[width=\textwidth]{figures/sapp/ISR/HT1080_GOT_DKO_TMRA_BAM15tit.pdf}
         \caption{BAM15, mito membrane potential}
         \label{fig:sapp:ISR:HT1080_GOT_DKO_TMRA_BAM15tit}
     \end{subfigure}
     \hfill
     \begin{subfigure}[b]{0.85\textwidth}
         \includegraphics[width=\textwidth]{figures/sapp/ISR/HT1080_GOT_DKO_TMRA_ASPtit.pdf}
         \caption{Asp depletion, mito membrane potential}
         \label{fig:sapp:ISR:HT1080_GOT_DKO_TMRA_ASPtit}
     \end{subfigure}
     \hfill
        \caption[Mito membrane potential in GOT DKO]{
        Mitochondrial membrane potential increases slightly during aspartate depletion in HT1080 GOT DKO.
        Measured on an Incucyte using the TMRE dye.
        }
        \label{fig:sapp:ISR:HT1080_GOT_DKO_TMRE}
\end{figure}











\FloatBarrier
\subsection{ASNS over-expression to ablate rotenone/antimycin induced ISR}
If asparagine depletion is the main reason for rotenone/antimycin induced ISR maybe over-expression of ASNS would mitigate it?
Using the HT1080 ATF4 reporter cells (low baseline clone) ASNS and eGFP (control) was over-expressed using lentiviral infection.
The polyclonal population that grew out after selection was tested for ATF4 upregulation using reporter assay and western blot.
Generally, ASNS over-expression diminish ATF4 upregulation, but strangely, in this background, adding additional asparagine increase ATF4.

\begin{figure}[ht]
     \centering
     \begin{subfigure}[b]{0.49\textwidth}
         \includegraphics[width=\textwidth]{figures/sapp/ISR/HT1080_ATF4rep_eGFP_OE.pdf}
         \caption{ATF4 reporter, eGFP polyclonal}
         \label{fig:sapp:ISR:HT1080_ATF4rep_eGFP_OE}
     \end{subfigure}
     \hfill
     \begin{subfigure}[b]{0.49\textwidth}
         \includegraphics[width=\textwidth]{figures/sapp/ISR/HT1080_ATF4rep_ASNS_OE.pdf}
         \caption{ATF4 reporter, ASNS polyclonal}
         \label{fig:sapp:ISR:HT1080_ATF4rep_ASNS_OE}
     \end{subfigure}
     \hfill
     \begin{subfigure}[b]{0.6\textwidth}
         \centering
         \includegraphics[width=\textwidth]{figures/sapp/ISR/HT1080_ISR_ASNS_OE.pdf}
         \caption{Western blot, eGFP/ASNS polyclonal}
         \label{fig:sapp:ISR:HT1080_ISR_ASNS_OE}
     \end{subfigure}
        \caption[ASNS OE, rotenone/antimycin induced ISR]{
        Effect of ASNS over-expression (eGFP control) on ATF4 reporter assay of ETC inhibitor induced ISR (parental HT1080 low baseline clone).
        For ATF4 reporter treatments was initiated by 10x spike-in, for western blot fresh media was added with drug at time zero.
        Using 100 nM rotenone and 1 µM antimycin.
        }
        \label{fig:sapp:ISR:ASNS_ISR}
\end{figure}







\FloatBarrier
\subsection{GCN2 relation to rotenone/antimycin induced ISR}
Now, one would think that ETC inhibitor induced ISR is induced through Asp/Asn depletion and thus mediated by GCN2 and thus a GCN2 knockout should not upregulate ATF4.
This model is what is proposed by Mick et al. \cite{Mick2020-kf} but it is only supported by one western blot showing GCN2 phosphorylation and a few experiment showing ablation of ATF4 induction when co-treating with 500 nM GCN2iB (Takeda \cite{Nakamura2018-mt}).
GCN2iB, is an ATP analog and there is conflicting evidence of its off-target tendencies.
The inventors, Nakamura et al. \cite{Nakamura2018-mt}, screens for off-target kinase inhibition but does not report specifically on related kinase HRI, PKR and PERK.
However, an AACR poster [\href{https://rapt.com/wp-content/uploads/2019/04/FLX-Bio-GCN2-poster-AACR-2019.pdf}{link}], from a competing group with their own drug \cite{Jackson2022-wv}, suggests that this and other GCN2 inhibitors have off-target effects on HRI with IC50 in the 10-500 nM range.
Thus, using 500 nM GCN2i as in Mick et al., and 2000 nM as we have done, could lead to robust GCN2 and HRI dual inhibition and mask any effect from HRI.
A GCN2 knockout would be a better way to differentiate the two.

We only have a confirmed GCN2 knockout in 293T cells and therefore tried with these.
Strangely, ISR in 293T cells was not reverted well with Asp/Asn, but even more strange the GCN2 KO cells did not have ablated ATF4.

\begin{figure}[ht]
    \centering
    \includegraphics[width=0.70\textwidth]{figures/sapp/ISR/293T_GCN2_ISR.pdf}
    \caption[ATF4 post mito inhib. GCN2 KO, western]{
    Rotenone treatment was initiated by adding fresh media with 100 nM rotenone at time zero.
    Rescue conditions: Asn (500 µM), Asp (30 mM) or Pyr (2 mM) was also added to media >1 h before time zero.
    Top, WT with different rescues.
    Bottom WT vs. GCN2 KO.
    }
    \label{fig:sapp:ISR:293T_GCN2_ISR}
\end{figure}


\FloatBarrier
We have pooled knockout cells (parental 143B ATF4 reporter high clone) of GCN2 that also maintain Asn rescuable ATF4 induction upon rotenone/antimycin treatment.
Monoclonal knockouts were never isolated due to the difficulty of GCN2 detection on western blots; however, the knockout protocol is typically very efficient.
Taking these data at face value, GCN2 is not required for rotenone/antimycin induced ISR.

\begin{figure}[ht]
    \centering
    \includegraphics[width=0.55\textwidth]{figures/sapp/ISR/143B_GCN2_ISR.pdf}
    \caption[ATF4 post mito inhib. GCN2 KO, reporter]{
    ATF4 reporter assay measured 21 h after drug treatment with vehicle, rotenone (100 nM) or antimycin (1 µM) spiked-in as 10x.
    }
    \label{fig:sapp:ISR:143B_GCN2_ISR}
\end{figure}






\FloatBarrier
\section{tRNA charge changes}
Supposedly, GCN2 phosphorylation is stimulated by uncharged tRNAs \cite{Wek1989-yw, Dong2000-si}, or ribosome collisions \cite{Harding2019-kb, Wu2020-lq, Yan2021-yv}, the latter explaining GCN2 activation upon UV radiation.
Measuring the tRNA charge effect of ISR inducing, aspartate depleting treatments, such as electron chain inhibitors, would be a good confirmation of whether ISR is driven by GCN2.

First, in 143B cell a high dose antimycin is spiked-in and tRNA charge followed over time (figure \ref{fig:sapp:tRNA:143B_Anti_time}).
Cytoplasmic tRNA\textsuperscript{Asp} remain charged throughout the time-course, but cytoplasmic tRNA\textsuperscript{Asn} starts to decrease after $\sim$9 hours.
Of note, the discontinuity of cyto-tRNA\textsuperscript{Asn} charge between 12 and 15 hours is likely due to sampling artifacts.
For practical reasons the 12 hour timepoint was spiked at 8:00 and harvested at 20:00 whereas the 15 hour timepoint was spiked at 19:00 and harvested the next day at 10:00.
This time difference would likely lead to higher media asparagine concentration due to cellular efflux, which would would then buffer the decrease in cyto-tRNA\textsuperscript{Asn} charge.
Regardless, mito-tRNA\textsuperscript{Asn} charge is rapidly decreasing in a timeframe that is compatible with the induction of ISR that happens within 1-2 hours.
Decreased charge of cyto-tRNA\textsuperscript{Asn} is not compatible with the early induction of ISR but it is with a later induction.
Thus, ISR may be induced two-fold - first an early GCN2 independent induction, and then a GCN2 dependent induction caused by uncharged cyto-tRNA\textsuperscript{Asn}.





The resulting eIF2alpha phosphorylation causes a global decrease in protein synthesis
Glu codon
.




\begin{figure}[!ht]
     \centering
     \begin{subfigure}[b]{0.6\textwidth}
         \includegraphics[width=\textwidth]{figures/sapp/tRNA/143B_Anti-time_Asp-Asn.pdf}
     \end{subfigure}
     \begin{subfigure}[b]{0.7\textwidth}
         \vspace{5pt}
         \includegraphics[width=\textwidth]{figures/sapp/tRNA/143B_Anti-time_Glu.pdf}
     \end{subfigure}
     \hfill
        \caption[Antimycin time-series in 143B, effect on tRNA charge]{
        tRNA charge as a function of time after antimycin treatment (1 µM spike-in) of 143B cells in DMEM, without pyruvate, with dialyzed FBS and 200 µM uridine.
        Top panel, effect on aspartate and asparagine tRNA charge.
        Bottom panel, effect on cytoplasmic glutamate codons with TTC charge being a potential marker inversely correlated with translation.
        For other tRNAs (cytoplasmic as well as mitochondrial) antimycin inhibitors had little or insignificant effect on charge.
        }
        \label{fig:sapp:tRNA:143B_Anti_time}
\end{figure}




\begin{figure}[!ht]
     \centering
     \begin{subfigure}[b]{0.6\textwidth}
         \includegraphics[width=\textwidth]{figures/sapp/tRNA/143B_ETCinhib_Asp-Asn.pdf}
     \end{subfigure}
     \begin{subfigure}[b]{0.8\textwidth}
         \vspace{5pt}
         \includegraphics[width=\textwidth]{figures/sapp/tRNA/143B_ETCinhib_Glu-Pro.pdf}
     \end{subfigure}
     \hfill
        \caption[ETC inhibitor in 143B, effect on tRNA charge]{
        tRNA charge after 30 hours treatment with vehicle, rotenone (50 nM), atpenin (5 µM), antimycin (0.5 µM) or oligomycin (0.25 µM).
        For all treatments cells were grown in DMEM, without pyruvate, with dialyzed FBS.
        For antimycin and oligomycin treatments, 200 µM uridine was added to the media.
        For the atpenin treatment, 1 mM pyruvate was added to the media.
        For other tRNAs (cytoplasmic as well as mitochondrial) ETC inhibitors had little or insignificant effect on charge.
        }
        \label{fig:sapp:tRNA:143B_ETCinhib}
\end{figure}





\begin{figure}[!ht]
     \centering
     \begin{subfigure}[b]{0.6\textwidth}
         \includegraphics[width=\textwidth]{figures/sapp/tRNA/H1299_ETCinhib_Asp-Asn.pdf}
     \end{subfigure}
     \begin{subfigure}[b]{0.8\textwidth}
         \vspace{5pt}
         \includegraphics[width=\textwidth]{figures/sapp/tRNA/H1299_ETCinhib_Glu-Pro.pdf}
     \end{subfigure}
     \hfill
        \caption[ETC inhibitor in H1299, effect on tRNA charge]{
        tRNA charge after 30 hours treatment with vehicle, rotenone (100 nM), atpenin (5 µM), antimycin (5 µM) or oligomycin (1 µM).
        For all treatments cells were grown in DMEM, without pyruvate, with dialyzed FBS.
        For antimycin and oligomycin treatments, 200 µM uridine was added to the media.
        For the atpenin treatment, 1 mM pyruvate was added to the media.
        For other tRNAs (cytoplasmic as well as mitochondrial) ETC inhibitors had little or insignificant effect on charge.
        }
        \label{fig:sapp:tRNA:H1299_ETCinhib}
\end{figure}











\begin{figure}[!ht]
     \centering
     \begin{subfigure}[b]{0.7\textwidth}
         \includegraphics[width=\textwidth]{figures/sapp/tRNA/143B-DKO_Asp-Asn.pdf}
     \end{subfigure}
     \begin{subfigure}[b]{0.7\textwidth}
         \vspace{5pt}
         \includegraphics[width=\textwidth]{figures/sapp/tRNA/H1299-DKO_charge_Asp-Asn.pdf}
     \end{subfigure}
     \begin{subfigure}[b]{0.7\textwidth}
         \vspace{5pt}
         \includegraphics[width=\textwidth]{figures/sapp/tRNA/HT1080-DKO_charge_Asp-Asn.pdf}
     \end{subfigure}
     \hfill
        \caption[tRNA charge in GOT DKO Asp-tit]{
        tRNA charge as a function of media aspartate concentration.
        Conditions with salvage mix (SM) contain: 500 µM asparagine, 200 µM uridine and 100 µM adenine.
        For other tRNAs (cytoplasmic as well as mitochondrial) aspartate titration had insignificant effect on charge.
        }
        \label{fig:sapp:tRNA:DKO_Asp-Asn}
\end{figure}


Put proliferation data here.




\begin{figure}[ht]
    \centering
    \includegraphics[width=0.65\textwidth]{figures/sapp/tRNA/H1299_GOT-DKO_prlfr.pdf}
    \caption[Asp titration in H1299 GOT DKO]{
    Proliferation assay with H1299 GOT DKO cells.
    Conditions with salvage mix (SM) contain: 500 µM asparagine, 200 µM uridine and 100 µM adenine.
    }
    \label{fig:sapp:ISR:H1299_DKO_prlfr}
\end{figure}


\begin{figure}[ht]
    \centering
    \includegraphics[width=0.65\textwidth]{figures/sapp/tRNA/HT1080_GOT-DKO_prlfr.pdf}
    \caption[Asp titration in H1080 GOT DKO]{
    Proliferation assay with HT1080 GOT DKO cells.
    Conditions with salvage mix (SM) contain: 500 µM asparagine, 200 µM uridine and 100 µM adenine.
    Made by David Sokolov.
    }
    \label{fig:sapp:ISR:HT1080_DKO_prlfr}
\end{figure}



\begin{figure}[ht]
    \centering
    \includegraphics[width=0.65\textwidth]{figures/sapp/ISR/ISRIB-Exp2.pdf}
    \caption[Effect of ISR inhibition on HT1080 GOT DKO]{
    Proliferation assay with HT1080 GOT DKO cells and ISRIB as an inhibitor of ISR.
    Made by David Sokolov.
    }
    \label{fig:sapp:ISR:HT1080_DKO_ISRIB}
\end{figure}





\FloatBarrier
\section{Metabolic changes post aspartate depletion in GOT DKO}
To get insights into what else is going on during aspartate depletion in GOT DKO metabolites were extracted post aspartate depletion for 143B (figure \ref{fig:sapp:GOT_DKO_Asp_depl:143B_DKO_metab}) and HT1080 (figure \ref{fig:sapp:GOT_DKO_Asp_depl:HT1080_DKO_metab}).
Some metabolic changes are readily explained: intracellular aspartate decrease drastically and rapidly.
This is also the case for asparagine in 143B cells but in HT1080 cells asparagine only decreases 50 percent.
Fumarate and malate are decreased, probably due to lower asp consumption in purine synthesis and thus lower fumarate production.
Similarly, the IMP to AMP ratio is increased in both cell lines, probably due to low aspartate preventing IMP to AMP synthesis.
For 143B the pyrimidines metabolites decrease as expected during aspartate shortage; however, strangely pyrimidines metabolites increase in HT1080.
Similarly, differential effect is seen for succinate and glycerol 3-phosphate.

\begin{figure}[!ht]
     \centering
     \begin{subfigure}[b]{0.45\textwidth}
         \includegraphics[width=\textwidth]{figures/sapp/GOT_DKO_Asp_depl/143B_DKO_AA.pdf}
         \caption{Amino acids}
         \label{fig:sapp:GOT_DKO_Asp_depl:143B_DKO_AA}
     \end{subfigure}
     \hspace{0.1\textwidth}
     \begin{subfigure}[b]{0.3\textwidth}
         \includegraphics[width=\textwidth]{figures/sapp/GOT_DKO_Asp_depl/143B_DKO_rd.pdf}
         \caption{Redox metabolites}
         \label{fig:sapp:GOT_DKO_Asp_depl:143B_DKO_rd}
     \end{subfigure}
     \hfill
     \begin{subfigure}[b]{0.6\textwidth}
         \includegraphics[width=\textwidth]{figures/sapp/GOT_DKO_Asp_depl/143B_DKO_tca.pdf}
         \caption{TCA metabolites}
         \label{fig:sapp:GOT_DKO_Asp_depl:143B_DKO_tca}
     \end{subfigure}
     \hfill
     \begin{subfigure}[b]{0.6\textwidth}
         \includegraphics[width=\textwidth]{figures/sapp/GOT_DKO_Asp_depl/143B_DKO_pyr.pdf}
         \caption{Pyrimidine metabolites}
         \label{fig:sapp:GOT_DKO_Asp_depl:143B_DKO_pyr}
     \end{subfigure}
     \hfill
     \begin{subfigure}[b]{0.3\textwidth}
         \includegraphics[width=\textwidth]{figures/sapp/GOT_DKO_Asp_depl/143B_DKO_pur.pdf}
         \caption{Purine metabolites}
         \label{fig:sapp:GOT_DKO_Asp_depl:143B_DKO_pur}
     \end{subfigure}
     \hfill
        \caption[Metabolic changes in 143B after Asp depl.]{
        Metabolic changes in 143B GOT DKO after aspartate depletion.
        Fold change compared to time zero 2 hours after media switch with/without aspartate.
        }
        \label{fig:sapp:GOT_DKO_Asp_depl:143B_DKO_metab}
\end{figure}


\begin{figure}[!ht]
     \centering
     \begin{subfigure}[b]{0.68\textwidth}
         \includegraphics[width=\textwidth]{figures/sapp/GOT_DKO_Asp_depl/HT1080_DKO_AA.pdf}
         \caption{Amino acids}
         \label{fig:sapp:GOT_DKO_Asp_depl:HT1080_DKO_AA}
     \end{subfigure}
     \hfill
     \begin{subfigure}[b]{0.45\textwidth}
         \includegraphics[width=\textwidth]{figures/sapp/GOT_DKO_Asp_depl/HT1080_DKO_rd.pdf}
         \caption{Redox metabolites}
         \label{fig:sapp:GOT_DKO_Asp_depl:HT1080_DKO_rd}
     \end{subfigure}
     \hfill
     \begin{subfigure}[b]{0.9\textwidth}
         \includegraphics[width=\textwidth]{figures/sapp/GOT_DKO_Asp_depl/HT1080_DKO_tca.pdf}
         \caption{TCA metabolites}
         \label{fig:sapp:GOT_DKO_Asp_depl:HT1080_DKO_tca}
     \end{subfigure}
     \hfill
     \begin{subfigure}[b]{0.9\textwidth}
         \includegraphics[width=\textwidth]{figures/sapp/GOT_DKO_Asp_depl/HT1080_DKO_pyr.pdf}
         \caption{Pyrimidine metabolites}
         \label{fig:sapp:GOT_DKO_Asp_depl:HT1080_DKO_pyr}
     \end{subfigure}
     \hfill
     \begin{subfigure}[b]{0.68\textwidth}
         \includegraphics[width=\textwidth]{figures/sapp/GOT_DKO_Asp_depl/HT1080_DKO_pur.pdf}
         \caption{Purine metabolites (and dTTP)}
         \label{fig:sapp:GOT_DKO_Asp_depl:HT1080_DKO_pur}
     \end{subfigure}
     \hfill
        \caption[Metabolic changes in HT1080 after Asp depl.]{
        Metabolic changes in HT1080 GOT DKO after aspartate depletion.
        Time-series of fold changes, normalized to vehicle at time zero.
        Condition with salvage mix (SM) contains: 1 mM Asn/Urd, 0.5 mM Hpx, 20 µM adenine and 10 µM adenosine.
        }
        \label{fig:sapp:GOT_DKO_Asp_depl:HT1080_DKO_metab}
\end{figure}





\FloatBarrier
\section{Conclusions regarding ISR}
General observation regarding the ISR experiments:
\begin{itemize}
    \item Using multiple cell lines (143B, HT1080, H1299, 293T) appear to confuse the conclusion, but it also reveals that ISR activation has a cell line dependent component.
    An example of this would be the difference between 143B and H1299 in their tRNA charge response after ETC inhibitor treatment (figure \ref{fig:sapp:tRNA:143B_ETCinhib} and \ref{fig:sapp:tRNA:H1299_ETCinhib}).
    \item The GCN2iB inhibitor is likely having an off-target effect on HRI which affects the conclusion from Mick et al. \cite{Mick2020-kf} and a few of our own western blots.
    Using lower a concentration may mitigate this, although care must be taken not to use a concentration so low that it causes an unintended GCN2 activation as described by Carlson et al. \cite{Carlson2023-zh}.
    \item Aspartate limitation induced by mitochondrial inhibitors in WT cell is difficult to compare to aspartate limitation in GOT DKO cells.
    This is particularly evident when looking at tRNA charge, but it has also been observed on western blots were asparagine rescues ETC inhibitor induced ISR in WT cell but not aspartate depletion induced ISR in GOT DKO cells.
    \item ETC inhibitor induced ISR is very sensitive to media conditions e.g. media change will increase the asparagine depletion by efflux.
    Media change can also by itself cause ATF4 induction which is reversible with media asparagine (figure \ref{fig:app_ch2:sal_frac_conc}).    
    \item Time scale is an important factor to consider in a model over the observations e.g. mito-tRNA\textsuperscript{Asn} becomes uncharged before cyto-tRNA\textsuperscript{Asn} in 143B cells treated with antimycin (figure \ref{fig:sapp:tRNA:143B_Anti_time}).
    \item Proline is also a redox active amino acid and mito-tRNA\textsuperscript{Pro} charge is severely affected by some ETC inhibitors.
    This is concordant with very low mitochondrial proline concentrations, $\sim$10 µM reported by Chen et al. \cite{Chen2016-mf}.
    \item Asparagine supplementation, in the form of salvage mix, rescues mito-tRNA\textsuperscript{Asp} charge to a different extend in different cell lines e.g. compared 143B and HT1080 in figure \ref{fig:sapp:tRNA:DKO_Asp-Asn}.
    Besides offsetting aspartate consumption, asparagine could be deaminated by glutaminase, possibly explaining the difference in charge rescue.
\end{itemize}



\subsection{Proposed model}
For WT cells treatment with ETC inhibitors results in a complex tRNA charge response that is cell line and drug dependent but typically leads to uncharged mito-tRNA\textsuperscript{Asp}, mito-tRNA\textsuperscript{Asn} and/or mito-tRNA\textsuperscript{Pro}.
For 143B and HT1080 cells the induction of ISR can be ablated with asparagine and OMA1/HRI knockout and while asparagine only partially ablates ISR in 293T cells, knockout of GCN2 is also insufficient.
Meanwhile, mito-tRNA\textsuperscript{Asn} becomes substantially uncharged after just 30 minutes while cyto-tRNA\textsuperscript{Asn} maintains full charge for 7 hours.
Therefore, I hypothesize the following:
Treatment with ETC inhibitors induces ISR through uncharged mitochondrial tRNAs of either mito-tRNA\textsuperscript{Asp}, mito-tRNA\textsuperscript{Asn} and/or mito-tRNA\textsuperscript{Pro}.
The resulting eIF2alpha phosphorylation causes a global decrease in protein synthesis and for some cell line/drug combination this fully prevents accumulation of uncharged cytoplasmic tRNAs (H1299) whereas for other cell line/drug combinations this only delays the aspartate/asparagine depletion, sparing cyto-tRNA\textsuperscript{Asp}, cyto-tRNA\textsuperscript{Asn} charge for some time before eventual depletion and a secondary GCN2 mediated ISR.
This leads to a testable predictions regarding cell line specific observations and the timing of ISR vs. uncharged tRNA.
For example OMA1/HRI knockout ablates ETC inhibitor induced ISR in HT1080, and thus these cells should harbour no uncharged cytoplasmic tRNA, while on the other hand since we know that 143B WT cell do eventually accumulate uncharged cytoplasmic tRNA they should have a GCN2 mediated ISR even with OMA1/HRI knockout.
Another testable prediction would be related to proline and its effect on ISR which could be tested by proline supplementation.

For all GOT DKO cell lines mito-tRNA\textsuperscript{Asp} charge decreases as a function of aspartate; however, we know from DARS2 KO and rho0 cells of multiple cell lines that neither mito-tRNA\textsuperscript{Asp} charge nor indeed any mitochondrial tRNAs are necessary for aspartate depletion induced ISR.
Therefore, I hypothesize the following:
Decreasing media aspartate to a level still compatible with cell proliferation induces ISR through uncharged mito-tRNA\textsuperscript{Asp}, the resulting eIF2alpha phosphorylation causes a global decrease in protein synthesis and therefore cyto-tRNA\textsuperscript{Asp} remains charged (figure \ref{fig:sapp:tRNA:DKO_Asp-Asn}).
Complete aspartate removal leads to the above but then shortly after cyto-tRNA\textsuperscript{Asp} becomes uncharged due to insufficient aspartate (supported by data in figure \ref{fig:sapp:ISR:HT1080_DKO_ASPtit_time}).
Similarly, this is what will happen in DARS2 KO and rho0 cells where the initial wave of ISR from uncharged mito-tRNA\textsuperscript{Asp} is blunted but the subsequent decrease in cyto-tRNA\textsuperscript{Asp} charge will lead to GCN2 mediated ISR.
Thus, the hypothesis would predict that blocking ISR, e.g. through ISRIB, or blocking uncharged mito-tRNA\textsuperscript{Asp} mediated ISR through rho0, would lead to cells with uncharged cyto-tRNA\textsuperscript{Asp} upon aspartate limitation.

How could uncharged mitochondrial tRNA cause ISR?
Maybe by mitochondrial proteotoxic stress.
In this hypothesis an acute increase of uncharged tRNAs would lead to stalling and fall off of mitochondrial ribosomes, unfinished and unfolded proteins would be released and this would be the basis of proteotoxic stress.
Indeed, I have observed upregulation of mitochondrial protease LONP1 which is thought to be involved in degrading unfolded or misfolded proteins.
More recently, Fessler et al. \cite{Fessler2022-ho} and Sekine et al. \cite{Sekine2023-qh} have implicated mitochondrial proteotoxic stress and LONP1 directly in the HRI mediated activation of the ISR.
They showed that over-expression induced mitochondrial proteotoxic stress induces ISR through DELE1 and that DELE1 is normally degraded by LONP1 in the matrix but upon iron depletion becomes stabilized on the outer membrane to activate HRI and subsequent ISR.








Way forward:

Use GOT DKO mito vs. rho0 cells to 
(figure \ref{fig:sapp:ISR:143B_DKO_ISR}, bottom panel)


More time points
Proline









\FloatBarrier
\section{GOT DKO characterization}


\begin{figure}[!ht]
     \centering
     \begin{subfigure}[b]{0.8\textwidth}
         \includegraphics[width=\textwidth]{figures/sapp/DKO_char/143B-WT-DKO_charge.pdf}
     \end{subfigure}
     \begin{subfigure}[b]{0.8\textwidth}
         \vspace{2pt}
         \includegraphics[width=\textwidth]{figures/sapp/DKO_char/H1299-WT-DKO_charge.pdf}
     \end{subfigure}
     \hfill
        \caption[tRNA charge in WT vs. GOT DKO]{
        tRNA charge in WT vs. DKO cell lines.
        For 143B GOT DKO, cells are in media with 20 mM Asp.
        For H1299 GOT DKO, cells are in media with 40 mM Asp.
        For other tRNAs (cytoplasmic as well as mitochondrial) GOT genotype had smaller or insignificant effect on charge.
        }
        \label{fig:sapp:tRNA:WT_vs_DKO}
\end{figure}


































 




\end{document}

\chapter*{Glossary}      % starred form omits the `chapter x'
\addcontentsline{toc}{chapter}{Glossary}
\thispagestyle{plain}
%
\begin{glossary}
\item[argument] replacement text which customizes a \LaTeX\ macro for
each particular usage.
\item[back-up] a copy of a file to be used when catastrophe strikes
the original.  People who make no back-ups deserve
no sympathy.
\item[control sequence] the normal form of a command to \LaTeX.
\item[delimiter] something, often a character, that indicates
the beginning and ending of an argument.
More generally, a delimiter is a field separator.
\item[document class] a file of macros that tailors \LaTeX\ for
a particular document.  The macros described by this thesis
constitute a document class.
\item[document option] a macro or file of macros
that further modifies \LaTeX\ for
a particular document.  The option {\tt[chapternotes]}
constitutes a document option.
\item[figure] illustrated material, including graphs,
diagrams, drawings and photographs.
\item[font] a character set (the alphabet plus digits
and special symbols) of a particular size and style.  A couple of fonts
used in this thesis are twelve point roman and {\sl twelve point roman
slanted}.
\item[footnote] a note placed at the bottom of a page, end of a chapter,
or end of a thesis that comments on or cites a reference
for a designated part of the text.
\item[formatter] (as opposed to a word-processor) arranges printed
material according to instructions embedded in the text.
A word-processor, on the other hand, is normally controlled
by keyboard strokes that move text about on a display.
\item[\LaTeX] simply the ultimate in computerized typesetting.
\item[macro]  a complex control sequence composed of 
other control sequences.
\item[pica] an archaic unit of length.  One pica is twelve points and
six picas is about an inch.
\item[point] a unit of length.  72.27 points equals one inch.
\item[roman]  a conventional printing typestyle using serifs.
the decorations on the ends of letter strokes.
This thesis is set in roman type.
\item[rule] a straight printed line; e.g., \hrulefill.
\item[serif] the decoration at the ends of letter strokes.
\item[table] information placed in a columnar arrangement.
\item[thesis] either a master's thesis or a doctoral dissertation.
This document also refers to itself as a thesis, although it
really is not one.
 
\end{glossary}